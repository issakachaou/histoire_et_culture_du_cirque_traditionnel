


\poemtitle{Lettre I}

\begin{verse}
	\small  
	Chère Tayssir,
	\hfill
	
	Amoureuse des couchers de soleil,\\
	Du bleu de la mer et du soleil vermeil,\\
	Elle est philosophe sans s’en rendre compte,\\
	À travers ses photographies elle le raconte.
	
	\smallskip
	
	Partout où ses yeux se sont posés,\\
	Elle ne perçut que du monde la beauté,\\
	Et dans sa manière d’être vivante,\\
	Je la trouve profondément inspirante.
	
	\smallskip
	
	Elle est belle par sa discrétion et sa classe,\\
	Et j’aime la découvrir, et le temps passe,\\
	Cette fille que je ne connaissais pas,\\
	Cette fille, philosophe d’Annaba.
	
	\smallskip
	
	Elle a une fleur de monoï dans les cheveux,\\
	Et, je trouve ça d’un charme merveilleux,\\
	Elle m’inspire cette fille qui va sûrement rougir,\\
	Quand je vais lui dire que j’aime l’appeler Tayssir.
\end{verse}
\newpage
\poemtitle{Lettre II}

\begin{verse}
	\small  
	Chère Tayssir,
	\hfill
	
	Amoureuse des couchers de soleil,\\
	Du bleu de la mer et du soleil vermeil,\\
	Elle est philosophe sans s’en rendre compte,\\
	À travers ses photographies elle le raconte.
	
	\smallskip
	
	Partout où ses yeux se sont posés,\\
	Elle ne perçut que du monde la beauté,\\
	Et dans sa manière d’être vivante,\\
	Je la trouve profondément inspirante.
	
	\smallskip
	
	Elle est belle par sa discrétion et sa classe,\\
	Et j’aime la découvrir, et le temps passe,\\
	Cette fille que je ne connaissais pas,\\
	Cette fille, philosophe d’Annaba.
	
	\smallskip
	
	Elle a une fleur de monoï dans les cheveux,\\
	Et, je trouve ça d’un charme merveilleux,\\
	Elle m’inspire cette fille qui va sûrement rougir,\\
	Quand je vais lui dire que j’aime l’appeler Tayssir.
\end{verse}




\thispagestyle{empty} %Dernière page vide
%\newpage
%\mbox{}
%\thispagestyle{empty}
