\chapter{Mes numéros préférés à Monte-Carlo et dans d’autres festivals}

\section*{\textit{Les tigres du Cirque Moira Orfei}}
\phantomsection
\addcontentsline{toc}{section}{Les tigres du Cirque Moira Orfei}

En 2004, à l’occasion de la 28\ieme~édition du festival international du cirque de Monte-Carlo, était présente l’une des plus fameuses familles circassiennes italiennes : la famille Orfei. En cette époque, Stefano Nones Orfei eut la lourde responsabilité de présenter le groupe de fauves qui devait performer pour cette édition. Ainsi, accompagné de sa troupe de danseuses et de sa femme, Anna Giurintano, il présenta les tigres du Cirque Moira Orfei.

Le numéro présenté lors de cette édition du festival était une libre interprétation d’un numéro de fauves qu’Alfred Court élabora pour le au Ringling Bros. and Barnum \& Bailey Circus en 1946. Court adapta le conte de Gabrielle-Suzanne de Villeneuve \textit{La Belle et la Bête} et proposa de son temps un numéro composé de douze léopards et d’une troupe de danseuses. Au départ, le numéro fut composé de douze danseuses, mais fut rapidement réduit de moitié.

La cage-ascenseur installée et les danseuses prêtes, sous le chapiteau monégasque, la lumière se tamisa et les tigres investirent la piste tour à tour. Sur la musique d’\textit{El tango de Roxanne} tirée du film \textit{Moulin Rouge}, Anna Giurintano s’éleva dans les airs et présenta un numéro de tissu aérien suspendu au-dessus de la cage aux fauves. La tension est à son comble lorsqu’elle décida d’exécuter sa figure finale. Tandis que Stefano fit cabrer l’ensemble des tigres, elle se suspendit tête la première et se laissa glisser d’un seul coup, ne laissant à son arrivée qu’un mètre entre elle et le sol. Sa prouesse réussit, elle sort de cage sous les applaudissements du public qui suit le rythme de la reprise de l’orchestre du festival.

Le numéro de Stefano peut alors débuter sous les merveilleux arrangements de l’orchestre du festival. Si l’honnêteté m’oblige à dire que les exercices présentés n’ont rien de novateur pour l’époque, ils eurent le mérite d’être maitrisés et présentés avec le panache à l’italienne du belluaire. Stefano joue avec ses fauves, les fait sauter, les fait rouler, les fait cabrer, en clair, il met en valeur les capacités de chacune de ses bêtes. Pour conclure son numéro, il fit cabrer une tigresse sur un tabouret orné d'une boule à facette qui tourna de sorte à faire un tour complet. L’image est belle, mais le meilleur est à venir. Pour finir, il fit marcher une autre tigresse sur les pattes arrière sur une distance qui équivaut à la moitié du diamètre de la piste. Sa dernière tigresse ayant quitté la cage, Stefano Nones Orfei fut rejoint par sa femme et ses danseuses pour saluer le public de Monte-Carlo conquis par cette véritable performance alliant numéro aérien et numéro de fauves.

Lors de cette édition du festival, il est important de noter que Stefano Nones Orfei présenta également un numéro d’animaux exotiques. Un numéro qui, malgré les applaudissements du public, n’apporta rien de très intéressant en terme technique. Le seul élément intéressant de ce numéro fut la présentation d’une autruche, d’un hippopotame et d’une girafe qui sont des animaux rarement présentés au cirque, mais malheureusement présentés de manière sommaire.

Pour son numéro de fauve novateur et sa mise en scène ancrée dans la modernité, il fut récompensé d’un clown d’argent.

\section*{\textit{La famille de Leonida Casartelli}}
\phantomsection
\addcontentsline{toc}{section}{La famille de Leonida Casartelli}

La famille Casartelli est l'une des plus célèbres dynasties de cirque italiennes et en 2007, elle entra dans l'histoire du cirque avec un des plus beaux numéros d'animaux exotiques que j'ai pu voir à l'heure actuelle. À l’occasion de la 31\ieme~édition du festival international du cirque de Monte-Carlo, la famille de Leonida Casartelli fut sélectionnée pour la présentation de deux numéros : un numéro de pas de deux à cheval et une pantomime orientale pour la présentation de leurs animaux exotiques. Si les deux numéros furent de qualité, dans cette section, j'aimerais m’intéresser plus particulièrement à leur tableau oriental.

À l’époque, ce numéro si unique fut présenté par Petit Gougou en ces mots : «\textit{ Mesdames et messieurs, le comité du festival international du cirque de Monte-Carlo est fier et heureux d’accueillir cette année une grande famille de cirque italien : la famille de Leonida Casartelli. Voici maintenant dans un tableau extraordinaire, fantastique. C’est un raz de marée humain qui va arriver maintenant dans cette piste et également une arche de Noé extraordinaire. Voici maintenant un conte merveilleux des Milles et une nuit : le conte d’Aladdin et la lampe merveilleuse racontée par Casartelli !} » Au cirque, il arrive parfois que le présentateur amplifie volontairement un numéro afin de mieux le vendre au public, mais cette fois-ci le tableau présenté fut à la hauteur de sa présentation. 

Ainsi, le numéro débuta par la musique \textit{Nuits d’Arabie}, écrite par Howard Ashman et composée par Alan Menken, chantée en direct par le conteur qui s’avançait progressivement vers la piste. Une ambiance orientale emplit le Chapiteau de l’Espace Fontevieille, et durant cette introduction, un mystérieux bédouin passa avec son dromadaire sur la piste, l’immersion était complète. Fait notable, l’histoire contée est plus fidèle à la version du film Disney de 1992 que du conte original. Le bédouin était Aladdin et après avoir fait son premier vœu, il fut transformé en riche Prince Ali. On découvrit un homme avec un splendide costume oriental orné de strass et de sequins, c’était Brian Casartelli, le dresseur. Dans son deuxième vœu, il demanda une belle princesse et voilà qu’apparut dans le rôle de la Princesse Jasmine, la belle Denise Sforzi, la cousine de Brian. Ensemble, ils entrèrent en piste et, sur une musique de fête, une dizaine de personnages les rejoignirent. Parmi eux, une jongleuse avec des torches en feux, un cracheur de feu et divers personnages que l’on pourrait voir dans un marché arabe du Moyen Âge. Soudain, les lumières orange et rouge du début laissèrent place au bleu de la nuit et Brian et Denise chantèrent ensemble \textit{Ce rêve bleu}, une nouvelle chanson tirée du film Aladdin de 1992. Durant cette chanson, ils montèrent sur un tapis volant créé pour le numéro et dès qu’ils furent installés dessus, il s’éleva dans les airs, le public applaudit d’étonnement la poésie des contes d’Orient. Lorsque le tapis les déposa, à la fin de la chanson, la piste fut vide. Le conteur continua son histoire en disant : « \textit{Le prince a voulu faire vraiment quelque chose de spécial pour sa Princesse alors, il lui a offert des diamants grands comme des chameaux !} » À ces mots, les cuivres de l’orchestre du festival entonnèrent de merveilleux arrangements orientaux qui accompagnèrent les huit danseuses sur la piste. 

C’est de cette manière que commença la présentation de la grande ménagerie de la famille Casartelli, sous la chambrière de Brian Casartelli. Au cours de ce véritable tableau exotique furent présentés successivement sur la piste des dromadaires qui furent ensuite rejoints par des lamas, puis un watusi et divers bovidés. Quel ne fut pas l’étonnement du public à l’arrivée d’un kangourou qui sauta au-dessus de chacun des cous de dromadaires allongés en bord de piste. Des émeus firent par la suite un bref tour de piste et regagnèrent la ménagerie. Bientôt, ils furent rejoints par les dromadaires qui quittèrent la piste, laissant la place à trois éléphants, un africain et deux indiens. Montés par trois danseuses d’une beauté sortie d’un conte des Mille et une nuits, les éléphants valsèrent et effectuèrent quelques exercices avant de quitter la piste à leur tour. L’étonnement du public fut à son apogée avec l’arrivée de deux girafes sur la piste. Il faut remettre ce numéro dans son contexte, nous sommes en 2007 et à l’époque, il est extrêmement rare de voir des girafes présentes en piste.

Après que les girafes eurent quitté la piste, le conteur conclut son conte et laissa place à la parade finale du Prince Ali et de la Princesse Jasmine qui arrivèrent en grande pompe. Sur la musique Prince Ali, Brian Casartelli et Denise Sforzi firent un tour de piste à cheval accompagnés par les servantes du palais, toutes munies d’oriflamme d’or. Le public applaudissait au rythme de la musique et Petit Gougou conclut : « \textit{Mesdames, mesdemoiselles, messieurs, ce conte merveilleux vous a été conté par la famille de Léonida Casartelli ! }» Par la suite, l’ensemble du chapiteau se leva pour acclamer le travail colossal qu'avait présenté la famille Casartelli. 

Ce numéro est sans nul doute mon numéro d’animaux exotiques préféré. Évidemment, lors d’autres éditions du festival, furent présentés d’autres numéros plus techniques et avec sans doute plus d’animaux. Cependant, ce qui m’a fasciné avec ce numéro et ce qui me fascinera toujours d’ailleurs, c'est la présentation d’une réelle pantomime d’une vingtaine de minutes. Tout a été pensé pour que l'immersion dans le conte d’\textit{Aladdin et la lampe merveilleuse} soit complète et ce fut un succès. Ce numéro me fait penser à ce qu’aurait pu être présenter fut une époque dans les grands cirques, je pense notamment à la famille Bouglione avec son spectacle \textit{La Perle du Bengale}. 

En cette année 2007, la famille Casartelli concourrait avec son tableau oriental et un numéro de pas de deux à cheval présenté par Brian et Ingrid Casartelli. La famille de Léonida Casartelli reçut lors de la remise des récompenses un clown d’or pour leur « \textit{Pas de deux à Cheval et l’ensemble de leurs présentations }». 

\section*{\textit{Steeve Eleky}}
\phantomsection
\addcontentsline{toc}{section}{Steeve Eleky}

Rares sont les clowns qui arrivent à me faire rire aux éclats à chaque fois. Steeve Eleky fait partie de ces clowns qui, même après avoir vu de nombreuses fois ses reprises, me font toujours autant rire. En 2012, à l’occasion de la 36\ieme~édition du festival international du cirque de Monte-Carlo, Steeve Eleky présenta ses deux reprises clownesques. 

Présenté dans une première entrée comme le remplaçant d’un jongleur hors du commun, il apparaît sans maquillage dans le public avec un kilt à carreaux. Très rapidement, le public se mit à rire à chacune de ses farces et comprit très rapidement qu’il était plus une arnaque ambulante qu’un jongleur talentueux. En effet, dans cette entrée, il promettait de jongler avec dix balles puis avec cinq cerceaux, mais trouva toujours le moyen astucieux de réussir en trichant avec son humour absurde. Il conclut ce numéro avec quelques passes à trois balles très bien exécutées et qui firent applaudir le public. 

Lors de sa seconde entrée, le clown absurde s’improvisa magicien dans un numéro hilarant. Le public en le voyant sut cette fois-ci qu'ils n’allaient pas voir un grand numéro de magie. Steeve Eleky n’avait plus l’effet de surprise pour l’aider, mais il réussit quand même à faire rire le public. Toujours dans son humour si caractéristique, il enchaina les petits tours de magie stupide. D’ailleurs, ce n’est même pas une attaque de ma part, il l’avoua lui-même à demi-mot durant son numéro. Ses tours étaient stupides, mais firent rire le public avec toujours autant de stupéfaction. Au fur et à mesure de ses tours de magie, il perdit progressivement son sérieux et pouffait de rire à chaque bêtise qu’il présenta au public. Pour conclure son numéro, il donna une leçon très intéressante sur la reproduction des lapins qui apparurent successivement d’une boite prétendue vide. Enfin, vide, jusqu’au moment où le magicien enleva une plaque cachant un double-fond, le tout en essayant de contenir son rire.

Cette année, Steeve Eleky sut conquérir le cœur du public et l’attention du jury qui le récompensa du prix spécial du Blackpool Tower Circus, du prix spécial du Cirque Nikulin Mouscou et du prix spécial Jérôme Medrano. 


\thispagestyle{empty} %Dernière page vide
%\newpage
%\mbox{}
%\thispagestyle{empty}

