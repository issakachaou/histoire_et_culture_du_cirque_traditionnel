\chapter{Mes numéros préférés à Monte-Carlo et dans d’autres festivals}

\section*{\textit{Les tigres du Cirque Moira Orfei}}
\phantomsection
\addcontentsline{toc}{section}{Les tigres du Cirque Moira Orfei}

En 2004, à l’occasion de la 28\ieme~édition du festival international du cirque de Monte-Carlo, était présente l’une des plus fameuses familles circassiennes italiennes : la famille Orfei. En cette époque, Stefano Nones Orfei eut la lourde responsabilité de présenter le groupe de fauves qui devait performer pour cette édition. Ainsi, accompagné de sa troupe de danseuses et de sa femme, Anna Giurintano, il présenta les tigres du Cirque Moira Orfei.

Le numéro présenté lors de cette édition du festival était une libre interprétation d’un numéro de fauves qu’Alfred Court élabora pour le au Ringling Bros. and Barnum \& Bailey Circus en 1946. Court adapta le conte de Gabrielle-Suzanne de Villeneuve \textit{La Belle et la Bête} et proposa de son temps un numéro composé de douze léopards et d’une troupe de danseuses. Au départ, le numéro fut composé de douze danseuses, mais fut rapidement réduit de moitié.

La cage-ascenseur installée et les danseuses prêtes, sous le chapiteau monégasque, la lumière se tamisa et les tigres investirent la piste tour à tour. Sur la musique d’\textit{El tango de Roxanne} tirée du film \textit{Moulin Rouge}, Anna Giurintano s’éleva dans les airs et présenta un numéro de tissu aérien suspendu au-dessus de la cage aux fauves. La tension est à son comble lorsqu’elle décida d’exécuter sa figure finale. Tandis que Stefano fit cabrer l’ensemble des tigres, elle se suspendit tête la première et se laissa glisser d’un seul coup, ne laissant à son arrivée qu’un mètre entre elle et le sol. Sa prouesse réussit, elle sort de cage sous les applaudissements du public qui suit le rythme de la reprise de l’orchestre du festival.

Le numéro de Stefano peut alors débuter sous les merveilleux arrangements de l’orchestre du festival. Si l’honnêteté m’oblige à dire que les exercices présentés n’ont rien de novateur pour l’époque, ils eurent le mérite d’être maitrisés et présentés avec le panache à l’italienne du belluaire. Stefano joue avec ses fauves, les fait sauter, les fait rouler, les fait cabrer, en clair, il met en valeur les capacités de chacune de ses bêtes. Pour conclure son numéro, il fit cabrer une tigresse sur un tabouret orné d'une boule à facette qui tourna de sorte à faire un tour complet. L’image est belle, mais le meilleur est à venir. Pour finir, il fit marcher une autre tigresse sur les pattes arrière sur une distance qui équivaut à la moitié du diamètre de la piste. Sa dernière tigresse ayant quitté la cage, Stefano Nones Orfei fut rejoint par sa femme et ses danseuses pour saluer le public de Monte-Carlo conquis par cette véritable performance alliant numéro aérien et numéro de fauves.

Lors de cette édition du festival, il est important de noter que Stefano Nones Orfei présenta également un numéro d’animaux exotiques. Un numéro qui, malgré les applaudissements du public, n’apporta rien de très intéressant en terme technique. Le seul élément intéressant de ce numéro fut la présentation d’une autruche, d’un hippopotame et d’une girafe qui sont des animaux rarement présentés au cirque, mais malheureusement présentés de manière sommaire.

Pour son numéro de fauve novateur et sa mise en scène ancrée dans la modernité, il fut récompensé d’un clown d’argent.

\section*{\textit{La famille de Leonida Casartelli}}
\phantomsection
\addcontentsline{toc}{section}{La famille de Leonida Casartelli}

La famille Casartelli est l’une des plus célèbres dynasties de cirque italienne et en 2007 elle entra dans l’histoire du cirque avec l’un des plus beaux numéros d’animaux exotiques que j’ai pu voir à l’heure actuelle. A l’occassion de la 31e édition du festival international du cirque de Monte-Carlo, la famille de Leonida Casartelli fut selectionnée pour la présentation de deux numéros : un numéro de pas de deux à cheval et une pantomime orientale pour la présentation de leurs animaux exotiques. Si les deux numéros furent de qualité, dans cette section j’aimerais m’intéresser plus particulièrement à leur tableau oriental.

A l’époque, ce numéro si unique fut présenté par Petit Gougou en ces mots : « \textit{Mesdames et messieurs, le comité du festival international du cirque de Monte-Carlo est fier et heureux d’accueillir cette année une grande famille de cirque italien : la famille de Leonida Casartelli. Voici maintenant dans un tableau extraordinaire, fantastique. C’est un raz de marée humain qui va arriver maintenant dans cette piste et également une arche de Noé extraordinaire. Voici maintenant un conte merveilleux des Milles et une nuit : le conte d’Aladdin et la lampe merveilleuse racontée par Casartelli !} » Au cirque, il arrive parfois que le présentateur amplifie volontairement un numéro afin de mieux le vendre au public, mais cette fois ci le tableau présenté fut à la hauteur de sa présentation. 

Ainsi le numéro débuta par la musique Nuits d’Arabie, écrite par Howard Ashman et composée Alan Menken, chanté en direct par le conteur qui s’avançait progressivement vers la piste. Une ambiance orientale emplit le Chapiteau de l’Espace Fontevieille, et durant cette introduction un mystérieux bédouin passa avec son dromadaire sur la piste, l’immersion était complète. Fait notable, l’histoire conté est plus fidèle la version du film disney de 1992 que du conte original. Le bédoin était Aladdin et après avoir fait son premier vœu, il fut transformé en riche Prince Ali. On découvrit un homme avec un splendide costume oriental orner de strass et de sequin, c’était Brian Casartelli, le dresseur. Dans son deuxième vœu il demanda une belle princesse et voila qu’apparu dans le rôle de Jasmine, la belle Denise Sforzi la cousine de Brian. Ensemble, ils entèrent en piste et sur musique de fête une dizaine de personnage les rejoignirent. Parmi une jongleuse avec de torche de feux, un cracheur de feu et divers personnages que l’on pourrait voir dans un marché arabe du Moyen Age. Soudain les lumières orange et rouge du début laissèrent place au bleu de la lui et Brian et Denise chantèrent ensemble Ce rêve bleu, une nouvelle chanson tirée du film Aladdin de 1992. Durant cette chanson, ils montèrent sur un tapis volant créé pour le numéro et dès qu’ils furent installé dessus il s’éleva dans les airs, le public applaudit d’étonnement la poésie des contes d’orient. Lorsque le tapis les déposa la fin de la chanson la piste vu vidé. Le conteur continua son histoire en disant : «\textit{ Le prince a voulu faire vraiment quelque chose de spécial pour sa princesse alors il lui a offert des diamants grands comme des chameaux !} »











