\chapter{Les hommes et les femmes qui font le cirque traditionnel}
\section*{\textit{Even Landri, le gladiateur des temps modernes}}\addcontentsline{toc}{section}{Even Landri, le gladiateur des temps modernes}
Even Landri est né le 2 août 1975 à Toulon dans une famille d'origine italienne et circassienne depuis plusieurs générations. La famille Landri est originaire de Torre Annunziata, une station balnéaire et thermale italienne à une vingtaine de kilomètres de Naples. Dès son enfance, Even vit dans une famille de cirque traditionnel qui pratique les différents arts de la piste et possède également de nombreux animaux sauvages. C'est dans cet environnement familial que rapidement, il côtoie de nombreux animaux, des plus communs aux plus atypiques. C'est dans cette enfance si particulière qu'il fut bercé par le rugissement des fauves, dont il se passionnera très rapidement alors qu'il n'avait que trois ans.

Il faut savoir que dans les familles circassiennes traditionnelles, les enfants s'initient souvent aux différents arts de la piste et présentent différents numéros au cours de leur vie. Durant leur jeunesse, ils prennent souvent plusieurs années pour trouver leur voie, le numéro qu'il leur plait particulièrement ou alors un numéro dans lequel il possède des facilités, leur spécialité. Trouver sa spécialité n'est donc pas simple face à la multitude de numéros possibles au cirque. Cependant, le cas des dresseurs de fauves reste assez particulier puisque les enfants qui se passionnent pour les fauves et entre en cage dès l'enfance en font le plus souvent leur voie, c'est le cas d'Even Landri. 

C'est grâce à son oncle Jean Landri que le jeune Even âgé de seulement 16 ans fera sa première entrée en cage en sa compagnie et qui par la suite lui apprendra tout l'art du dressage. C'est en répétant sous l'œil éclairé de son oncle qu’en 1992, à l'âge de 17 ans, il présentera son premier numéro composé à l'époque de trois tigres, numéro auquel il intégrera quatre autres tigres l'année suivante. Treize ans plus tard, en 2006, Even décide de monter un numéro mixte de fauves, c’est-à-dire composé de différentes espèces, ici de quatre tigres et de quatre lionnes, numéro qu'il présentera en public l'année suivante, une prouesse quand on sait qu'un numéro animalier prend habituellement plusieurs années à être monté. Durant mon enfance, c'est grâce à ce numéro que j'ai connu, l'homme que l'on surnomme encore aujourd'hui Le gladiateur des temps modernes. Quatre ans plus tard, en 2010, Even Landri intègre deux tigres blancs à son numéro mixte de fauves, désormais composé de quatre lionnes et de six tigres. Il est d'ailleurs important de noter que les tigres blancs ne représentent pas une race de tigre à part entière, ils représentent le plus souvent une variété de tigres du Bengale. 

Sa méthode pour dresser ses fauves commence dès leur naissance avec de l'éducation à l'âge de quatre mois avec la mère. Par la suite, il commencera ce que l'on peut considérer comme les bases du dressage à 10 mois. Comme la plupart des dresseurs, il travaille à la récompense, c'est-à-dire en morceaux de viande. Cette technique, bien connue du milieu du dressage, permet un apprentissage dans le jeu et le respect de l'animal. Even Landri est le seul dresseur de France à faire marcher un tigre sur ces pattes arrière sans chambrière ni bambou, ces deux instruments ne servant uniquement à donner des indications aux fauves sans jamais les toucher.

Actuellement, il finit chacun de ses numéros les mains nues, ce qui est exceptionnel dans le milieu du dressage de fauves et surtout extrêmement dangereux. Il travaille au Cirque de Venise, le cirque de sa famille dans lequel il présente tous les animaux du cirque excepté la cavalerie présentée par Steeve Landri, son frère. À l'heure actuelle, il monte un nouveau numéro mixte de fauves avec un tigre snow, une variété de tigres blancs qui ont la particularité d'avoir des rayures si claires que l'on pourrait croire qu'ils n'en possèdent pas.

J'ai connu Even Landri grâce à une vidéo de son numéro qui était sur internet. Il existe de nombreuses captations des numéros de fauves d'Even Landri, chacun de ses numéros sont d'ailleurs exceptionnels, c'est pourquoi Even Landri est considéré par ses pairs comme l'un des dresseurs de fauves les plus talentueux de l'époque. La vidéo qui me l'a fait découvrir a été tournée par Jon Notenboom le 25 octobre 2009 à Carpentras avec l'accord du dresseur. Cette vidéo est encore disponible 13 ans après sa mise en ligne. C'est l'un de mes dresseurs préférés, il est le parfait représentant d'un dressage traditionnel créatif, un dressage à l'italienne dont se dégage un panache éclatant. Je vous invite donc à voir ou revoir cette vidéo pour vous faire votre avis sur celui que l’on nomme encore aujourd'hui, et de manière justifiée, Le gladiateur des temps modernes.

\section*{\textit{Pierre Marchand, le virtuose du diabolo}}\addcontentsline{toc}{section}{Pierre Marchand, le virtuose du diabolo}



Pierre Mazieri, plus connu sous son nom de scène ``Pierre Marchand'', est né un 5 septembre à Saint-Mandé. Il est l'ainé d'une fratrie de trois enfants d'origine Corse. Pierre Marchand n'est pas né d'une famille circassienne, ses parents sont tous deux enseignants, rien ne le prédestinait donc à devenir l'artiste talentueux qu'il allait devenir. Bien qu'il soit né en France, il vivra une partie de son enfance au Togo grâce au déplacement professionnel de ses parents. En Afrique, il vivra une enfance heureuse et sera un enfant plein d'énergie. Malheureusement, quelque temps plus tard, à l'âge de ses 7 ans, ses parents doivent rentrer en France. Pierre vivra comme un déchirement l'abandon de sa terre presque natale.

Peu habitué à son nouvel environnement, l'Île-de-France, il vit mal au rythme de la ville et déborde vite d'énergie. Au fil du temps, ses parents peinent à le canaliser et en septembre, sa mère décide de l'inscrire à l'école nationale du cirque à Paris, dirigée par Annie Fratellini et Pierre Etaix. Aujourd'hui cette école n'existe plus, mais elle a été succédée par l'académie Annie Fratellini. C'est grâce à cette inscription que tous les mercredis et samedis après-midi Pierre se défoulait tout en apprenant également de nouvelles disciplines. Pierre se plait tellement dans ce nouvel univers qu'est le cirque et dans cette école qu'en décembre, il se fera remarquer par Annie Fratellini qui décèle en lui une future âme d'artiste. Elle décide alors de le prendre sous son aile afin de lui transmettre son savoir-faire : les bases du jonglage.

Ses après-midi, il les passe à l'école du cirque avec son maitre Italo Medini, un célèbre jongleur qui lui apprendra toutes les subtilités du diabolo. À partir de cet instant, Pierre aura une scolarité à horaires aménagés dans laquelle il pourra pleinement s'épanouir, le parfait mélange de l'école et du cirque. Pendant 9 ans, il tiendra ce rythme de trente heures de diabolo par semaine jusqu'à l'obtention de son bac scientifique. Par la suite il décide de lancer sa carrière avec son numéro qu’il vient de finir d'élaborer.
Son numéro qu'il présente encore aujourd'hui et le même depuis le début de sa carrière même s'il a dû monter des numéros plus condensés pour l'émission La France à un incroyable talent. Pendant les sept minutes de numéro qu'il propose, ce soleil corse nous donne de l'énergie pure à tel point qu'à chaque numéro, il perd 1,4 kg. C'est à cette époque qu'il décide de lancer sa carrière de jongleur professionnel et cette carrière va être propulsé par une rencontre en particulier celle de Vincent Lagaff.

Vincent Lagaff lui propose de présenter son numéro dans son émission Le Bigdil, et par la même occasion lance la carrière de celui qui se fera reconnaitre mondialement comme Pierre Marchand. C'est aussi à cette époque qu'il se produira dans de nombreux festivals de cirque à travers le monde. En 2004, il remporte par exemple la médaille d'or au festival de Wiesbaden en Allemagne et, en 2006 au festival du cirque de Budapest, le prix du Cirque de Moscou et le prix de la ville de Budapest. Lors de la 31e édition du festival international du cirque de Monte-Carlo, il reçoit une standing ovation du public et se voit décerner le prix du club du Cirque. La même année, en 2007, il remporte un Loyal d'or, la plus haute récompense décernée à l'occasion de la 2e édition du festival international du cirque de Bayeux.

Grâce à ses nombreuses récompenses, il arrive à travailler dans les établissements de spectacles les plus prestigieux du monde. À partir de 2006, il signe un contrat au Lido, l'un des plus célèbres cabarets parisiens, où il se produira pendant huit ans pour la revue Bonheur, avant de rejoindre le Cirque d'Hiver Bouglione pour la tournée événement Bravo. C'est durant cette période, en 2012, qu'il passera dans l'émission de télévision La France a un incroyable talent qui le fera connaitre.

L'un des candidats préférés du public, il arrivera en première place en demi-finale, mais hélas finira l'émission finale en 5e place. En 2016, il se présente au Moulin Rouge avant de partir l'année suivante pour le Danemark et la Suède. En 2018, il rentre en France pour répondre aux appels de Pierre Meyer, qui le sollicite depuis des années, et se présente au Royal Palace Kirrwiller. En 2019, après une brève escale aux Etats-Unis, il se produira pour l'Europa Park, le plus grand parc d'attraction d'Allemagne, puis il repart pour le Brésil, pour la Suisse. Évidemment, durant sa carrière, en plus de prestigieux cabarets, il se produit également dans des cirques de renom à travers le monde comme les Cirques Krone, Roncalli et Flic Flac en Allemagne, mais aussi le Cirque national norvégien, au Cirque Tihany au Mexique et lors de la tournée Excentrik du Cirque Arlette Grüss en 2021. Le 4 et 5 décembre 2021, il se présenta à Nantes pour le spectacle La H Arena fait son cirque, sous la direction artistique de Joseph Bouglione.

En tant qu'artiste jongleur, Pierre Marchand a toujours eu une place particulière dans mon cœur, puisque c'est lui qui me donna envie de faire du diabolo ma spécialité. Pour moi comme pour de nombreux passionnés, même si ce n’est pas le plus technique des jongleurs, il a toujours été au-dessus d'une certaine manière des autres diabolistes. Pendant des heures, je me rappelle encore apprendre tous ses gestes, ses mimiques faciales, son style si particulier qu'encore aujourd'hui, en 2023, je retrouve malgré moi dans ma manière de jongler. Il y aura toujours une part de Pierre Marchand en moi.
