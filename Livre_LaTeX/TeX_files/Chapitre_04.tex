\chapter{Les visages du cirque traditionnel}
\section*{\textit{Even Landri, le gladiateur des temps modernes}}
\phantomsection
\addcontentsline{toc}{section}{Even Landri, le gladiateur des temps modernes}

Even Landri est né le 2 août 1975 à Toulon dans une famille d'origine italienne et circassienne depuis plusieurs générations. La famille Landri est originaire de Torre Annunziata, une station balnéaire et thermale italienne à une vingtaine de kilomètres de Naples. Dès son enfance, Even vit dans une famille de cirque traditionnel qui pratique les différents arts de la piste et possède également de nombreux animaux sauvages. C'est dans cet environnement familial que rapidement, il côtoie de nombreux animaux, des plus communs aux plus atypiques. C'est dans cette enfance si particulière qu'il fut bercé par le rugissement des fauves, dont il se passionnera très rapidement alors qu'il n'avait que trois ans.

Il faut savoir que dans les familles circassiennes traditionnelles, les enfants s'initient souvent aux différents arts de la piste et présentent différents numéros au cours de leur vie. Durant leur jeunesse, ils prennent souvent plusieurs années pour trouver leur voie, le numéro qu'il leur plait particulièrement ou alors un numéro dans lequel il possède des facilités, leur spécialité. Trouver sa spécialité n'est donc pas simple face à la multitude de numéros possibles au cirque. Cependant, le cas des dresseurs de fauves reste assez particulier puisque les enfants qui se passionnent pour les fauves et entre en cage dès l'enfance en font le plus souvent leur voie, c'est le cas d'Even Landri. 

C'est grâce à son oncle Jean Landri que le jeune Even âgé de seulement 16 ans fera sa première entrée en cage en sa compagnie et qui par la suite lui apprendra tout l'art du dressage. C'est en répétant sous l'œil éclairé de son oncle qu’en 1992, à l'âge de 17 ans, il présentera son premier numéro composé à l'époque de trois tigres, numéro auquel il intégrera quatre autres tigres l'année suivante. Treize ans plus tard, en 2006, Even décide de monter un numéro mixte de fauves, c’est-à-dire composé de différentes espèces. Ce numéro composé de quatre tigres et de quatre lionnes, il le présenta en public seulement un an après son élaboration, une prouesse quand on sait qu'un numéro animalier prend habituellement plusieurs années à être monté. Durant mon enfance, c'est grâce à ce numéro que j'ai connu, l'homme que l'on surnomme encore aujourd'hui \textit{Le gladiateur des temps modernes}. Quatre ans plus tard, en 2010, Even Landri intègre deux tigres blancs à son numéro mixte de fauves, désormais composé de quatre lionnes et de six tigres. Il est d'ailleurs important de noter que les tigres blancs ne représentent pas une race de tigre à part entière, ils représentent le plus souvent une variété de tigres du Bengale. 

Sa méthode pour dresser ses fauves commence dès leur naissance avec de l'éducation à l'âge de quatre mois avec la mère. Par la suite, il commencera ce que l'on peut considérer comme les bases du dressage à 10 mois. Comme la plupart des dresseurs, il travaille à la récompense, c'est-à-dire en morceaux de viande. Cette technique, bien connue du milieu du dressage, permet un apprentissage dans le jeu et le respect de l'animal. Even Landri est le seul dresseur de France à faire marcher un tigre sur ces pattes arrière sans chambrière ni bambou, ces deux instruments ne servant uniquement à donner des indications aux fauves sans jamais les toucher.

Actuellement, il finit chacun de ses numéros les mains nues, ce qui est exceptionnel dans le milieu du dressage de fauves et surtout extrêmement dangereux. Il travaille au Cirque de Venise, le cirque de sa famille dans lequel il présente tous les animaux du cirque excepté la cavalerie présentée par Steeve Landri, son frère. À l'heure actuelle, il monte un nouveau numéro mixte de fauves avec un tigre snow, une variété de tigres blancs qui ont la particularité d'avoir des rayures si claires que l'on pourrait croire qu'ils n'en possèdent pas.

J'ai connu Even Landri grâce à une vidéo de son numéro qui était sur internet. Il existe de nombreuses captations des numéros de fauves d'Even Landri, chacun de ses numéros sont d'ailleurs exceptionnels, c'est pourquoi Even Landri est considéré par ses pairs comme l'un des dresseurs de fauves les plus talentueux de l'époque. La vidéo qui me l'a fait découvrir a été tournée par Jon Notenboom le 25 octobre 2009 à Carpentras avec l'accord du dresseur. Cette vidéo est encore disponible 13 ans après sa mise en ligne. C'est l'un de mes dresseurs préférés, il est le parfait représentant d'un dressage traditionnel créatif, un dressage à l'italienne dont se dégage un panache éclatant. Je vous invite donc à voir ou revoir cette vidéo pour vous faire votre avis sur celui que l’on nomme encore aujourd'hui, et de manière justifiée, \textit{Le gladiateur des temps modernes}.

\section*{\textit{Pierre Marchand, le virtuose du diabolo}}
\phantomsection
\addcontentsline{toc}{section}{Pierre Marchand, le virtuose du diabolo}

Pierre Mazieri, plus connu sous son nom de scène ``Pierre Marchand'', est né un 5 septembre à Saint-Mandé. Il est l'ainé d'une fratrie de trois enfants d'origine Corse. Pierre Marchand n'est pas né d'une famille circassienne, ses parents sont tous deux enseignants, rien ne le prédestinait donc à devenir l'artiste talentueux qu'il allait devenir. Bien qu'il soit né en France, il vivra une partie de son enfance au Togo grâce au déplacement professionnel de ses parents. En Afrique, il vivra une enfance heureuse et sera un enfant plein d'énergie. Malheureusement, quelque temps plus tard, à l'âge de ses 7 ans, ses parents doivent rentrer en France. Pierre vivra comme un déchirement l'abandon de sa terre presque natale.

Peu habitué à son nouvel environnement, l'Île-de-France, il vit mal au rythme de la ville et déborde vite d'énergie. Au fil du temps, ses parents peinent à le canaliser et en septembre, sa mère décide de l'inscrire à l'école nationale du cirque à Paris, dirigée par Annie Fratellini et Pierre Etaix. Aujourd'hui cette école n'existe plus, mais elle a été succédée par l'académie Annie Fratellini. C'est grâce à cette inscription que tous les mercredis et samedis après-midi Pierre se défoulait tout en apprenant également de nouvelles disciplines. Pierre se plait tellement dans ce nouvel univers qu'est le cirque et dans cette école qu'en décembre, il se fera remarquer par Annie Fratellini qui décèle en lui une future âme d'artiste. Elle décide alors de le prendre sous son aile afin de lui transmettre son savoir-faire : les bases du jonglage.

Ses après-midi, il les passe à l'école du cirque avec son maitre Italo Medini, un célèbre jongleur qui lui apprendra toutes les subtilités du diabolo. À partir de cet instant, Pierre aura une scolarité à horaires aménagés dans laquelle il pourra pleinement s'épanouir, le parfait mélange de l'école et du cirque. Pendant 9 ans, il tiendra ce rythme de trente heures de diabolo par semaine jusqu'à l'obtention de son bac scientifique. Par la suite il décide de lancer sa carrière avec son numéro qu’il vient de finir d'élaborer.

Son numéro qu'il présente encore aujourd'hui et le même depuis le début de sa carrière même s'il a dû monter des numéros plus condensés pour l'émission La France à un incroyable talent. Pendant les sept minutes de numéro qu'il propose, ce soleil corse nous donne de l'énergie pure à tel point qu'à chaque numéro, il perd 1,4 kg. C'est à cette époque qu'il décide de lancer sa carrière de jongleur professionnel et cette carrière va être propulsé par une rencontre en particulier celle de Vincent Lagaff.

Vincent Lagaff lui propose de présenter son numéro dans son émission Le Bigdil, et par la même occasion lance la carrière de celui qui se fera reconnaitre mondialement comme Pierre Marchand. C'est aussi à cette époque qu'il se produira dans de nombreux festivals de cirque à travers le monde. En 2004, il remporte par exemple la médaille d'or au festival de Wiesbaden en Allemagne et, en 2006 au festival du cirque de Budapest, le prix du Cirque de Moscou et le prix de la ville de Budapest. Lors de la 31\ieme~édition du festival international du cirque de Monte-Carlo, il reçoit une standing ovation du public et se voit décerner le prix du club du Cirque. La même année, en 2007, il remporte un Loyal d'or, la plus haute récompense décernée à l'occasion de la 2\ieme~édition du festival international du cirque de Bayeux.

Grâce à ses nombreuses récompenses, il arrive à travailler dans les établissements de spectacles les plus prestigieux du monde. À partir de 2006, il signe un contrat au Lido, l'un des plus célèbres cabarets parisiens, où il se produira pendant huit ans pour la revue Bonheur, avant de rejoindre le Cirque d'Hiver Bouglione pour la tournée événement \textit{Bravo}. C'est durant cette période, en 2012, qu'il passera dans l'émission de télévision La France a un incroyable talent qui le fera connaitre.

L'un des candidats préférés du public, il arrivera en première place en demi-finale, mais hélas finira l'émission finale en 5\ieme~place. En 2016, il se présente au Moulin Rouge avant de partir l'année suivante pour le Danemark et la Suède. En 2018, il rentre en France pour répondre aux appels de Pierre Meyer, qui le sollicite depuis des années, et se présente au Royal Palace Kirrwiller. En 2019, après une brève escale aux Etats-Unis, il se produira pour l'Europa Park, le plus grand parc d'attraction d'Allemagne, puis il repart pour le Brésil, pour la Suisse. Évidemment, durant sa carrière, en plus de prestigieux cabarets, il se produit également dans des cirques de renom à travers le monde comme les Cirques Krone, Roncalli et Flic Flac en Allemagne, mais aussi le Cirque national norvégien, au Cirque Tihany au Mexique et lors de la tournée \textit{Excentrik} du Cirque Arlette Grüss en 2021. Le 4 et 5 décembre 2021, il se présenta à Nantes pour le spectacle La H Arena fait son cirque, sous la direction artistique de Joseph Bouglione.

En tant qu'artiste jongleur, Pierre Marchand a toujours eu une place particulière dans mon cœur, puisque c'est lui qui me donna envie de faire du diabolo ma spécialité. Pour moi comme pour de nombreux passionnés, même si ce n’est pas le plus technique des jongleurs, il a toujours été au-dessus d'une certaine manière des autres diabolistes. Pendant des heures, je me rappelle encore apprendre tous ses gestes, ses mimiques faciales, son style si particulier qu'encore aujourd'hui, en 2023, je retrouve malgré moi dans ma manière de jongler. Il y aura toujours une part de Pierre Marchand en moi.

\section*{\textit{Martin Lacey Junior, l'homme au clown d'or}}
\phantomsection
\addcontentsline{toc}{section}{Martin Lacey Junior, l'homme au clown d'or}

Martin Lacey Junior est né le 8 juin 1977 à Sunderland au Royaume-Uni, dans une fratrie de trois enfants. Ses parents sont directeurs de zoo et dresseurs d’animaux. En effet, durant de nombreuses années, son père Martin Lacey Sr., également directeur du Big Apple Circus, travaillait avec un groupe de lions tandis que sa mère Susan travaillait avec un groupe mixte de tigres et de lions. Susan Lacey a d'ailleurs reçu un clown d'argent en 2005 pour la présentation de son groupe de tigres blancs, tandis que son fils, Alexander, le frère aîné de Martin, reçu un clown d'argent pour la présentation de son groupe mixte en 2003. On peut donc voir que la spécialité des Lacey est le dressage de fauves. Dès la naissance, Martin vivra dans cet environnement dans lequel il sera au plus proche de nombreux animaux. Très vite, il se passionnera pour les fauves avec lesquelles il rêve secrètement de travailler un jour.

Comme de nombreux enfants circassiens, il suit dans un premier temps une éducation itinérante, mais en 1988, les parents du jeune Martin, âgé à l’époque de 11 ans, sont de plus en plus soucieux de lui inculquer un bon niveau d'instruction et décident de l'inscrire en pensionnat. Il fera donc une partie de sa scolarité à la Cordeaux Highschool dans le comté du Lincolnshire, dans la région des Midlands de l'Est en Angleterre. C'est durant cette période de sa scolarité qu'il se passionnera pour de nouvelles disciplines comme le rugby et la boxe.

En 1994, Martin a désormais 17 ans et après avoir obtenu son baccalauréat, il ne pense qu'à une seule chose : revenir au cirque, retrouver sa famille et ses animaux. Cependant, son père décide alors de mettre ses études à profit et le délègue au service marketing. Martin aime le domaine de la publicité dans lequel il se plait, mais rapidement l'appel des fauves devient trop fort, et il choisit de se lancer en tant que dresseur de fauves et marche dans les pas de la tradition familiale.

Il rejoint donc son frère Alexander et monte son premier numéro mixte de fauves. Martin Lacey Junior fera alors ses débuts au Cirque Pauwels et au Cirque Kino's, avant de rejoindre le Cirque Krone. Trois ans plus tard, Martin, désormais âgé de 20 ans, travaille seul son numéro de lion avec lequel il se fera connaitre notamment au Cirque Krone. Il y posera d’ailleurs ses valises après s'être marié avec Jana Madana la fille de Christel Sembach-Krone, la directrice du Cirque Krone, l'un des plus grands cirques d'Allemagne.

En 1999, il se présente au festival international du cirque de Massy avec son numéro composé à l'époque de lionnes et remporte un chapiteau de Cristal. En janvier 2000, il remporte un clown d'argent avec son groupe de lionnes lors de la 20\ieme~édition du festival international du cirque de Monte-Carlo. En 2004, il remporte une étoile d'or au festival international du cirque Auvergne Rhône-Alpes Isère. En 2010, c'est la consécration pour Martin qui remporte à 33 ans un clown d'or pour son numéro composé de lionnes et d’un lion. En 2019, il reproduit l'exploit et obtient de nouveau un clown d'or pour son numéro et par la même occasion entre définitivement dans l'histoire. Il est également important de noter qu'en 2016, pour la 40\ieme~édition du festival international du cirque de Monte-Carlo, Martin Lacey Junior partage la piste avec des légendes du dressage de fauve que sont Nicolaï Pavlenko et Massimiliano Nones.

Grâce à ses nombreuses récompenses qui feront sa carrière, il se produira à travers le monde, notamment pour le Cirque d'Hiver Bouglione en 2011 lors du gala \textit{La Perle du Bengale} au Bourget. Martin Lacey Junior a pour objectif de prouver aux yeux du monde que le dressage de fauves est un art en organisant régulièrement des entrainements ouverts au public, afin de montrer la manière dont il travaille.

Martin Lacey Junior a longtemps été l'un de mes dresseurs de fauves préférés, aux côtés de Frédéric Edelstein et de Steeve Caplot, c'est l'exemple même du dressage moderne et en douceur. Si vous deviez retenir uniquement deux de ses numéros, je vous conseille de voir ses prestations lors de la 30\ieme~et de la 34\ieme~édition du festival international du cirque de Monte-Carlo.

\section*{\textit{Banbino Mouredon, et le baiser de la mort}}
\phantomsection
\addcontentsline{toc}{section}{Banbino Mouredon, et le baiser de la mort}

Banbino Mouredon est né le 5 septembre 1940 au Grand-Bourg, à une vingtaine de kilomètres de Guéret dans la Creuse. Sa famille, les Mouredon, est originaire du Gard dans le sud de la France et circassienne depuis sept générations.

Banbino est connu pour avoir une grande carrière de dresseur de fauve. Il se fera notamment connaitre pour être l'un des rares dresseurs de fauves français de l'époque à présenter le baiser de la mort avec un lion. En parallèle de sa carrière de dresseur, il sera également directeur de son cirque familial et tournera sous différentes enseignes comme le Mondial Circus, le Cirque Annie Fratellini ou encore le fameux Cirque Achille Zavatta

En 2018, à l'âge de 78 ans, il se présente pour la dernière fois sur une piste de cirque à Neuilly-sur-Marne en Seine-Saint-Denis. Le 5 septembre 2022, il fêta ses 83 ans, toujours avec le sourire et heureux, le vieux lion est désormais à la retraite.

\section*{\textit{Tom Dieck Junior, l'homme qui parlait à l'oreille des ligrons}}
\phantomsection
\addcontentsline{toc}{section}{Tom Dieck Junior, l'homme qui parlait à l'oreille des ligrons}

Tom Dieck Junior est né le 30 avril 1983 à Montbrison, à une quarantaine de kilomètres de Saint-Etienne, dans la Loire. Il est le fils de Tom et de Gilian Dieck, de célèbres dresseurs de fauves, mais aussi le petit-fils de Tom Dieck Senior. Il fait partie de la famille Dieck, une famille circassienne depuis plusieurs générations, c'est donc tout naturellement que le fils voulu faire comme le père, comme le père avait voulu faire comme son père avant lui, afin de perpétuer la tradition familiale.

En décembre 2003, alors qu'il n’a que 21 ans, il met en place un numéro mixte de fauves et le présente au Cirque Jules Verne à Amiens, c'est le début d'une carrière fulgurante qui s'annonce pour Tom Dieck Junior. Deux ans plus tard, il rejoint la maison Arlette Grüss pour leur création \textit{Rêves}, qui fête le vingtième anniversaire d'existence du cirque, avant de s'exporter l'année suivante en Allemagne pour se présenter au Cirque Probst. C'est à cette époque qu'il remporte le Muermans-Vastgoed Circus Award.

L'année suivante en 2006, il présente son groupe mixte de fauves lors de la 14\ieme~édition du festival international du cirque de Massy et remporte un chapiteau de cristal. La même année, il remporte le 2\ieme~prix du jury à l'occasion du 11\ieme~festival du cirque d'Enschede aux Pays-Bas. En 2007, à l'occasion de la 31\ieme~édition du festival international du cirque de Monte-Carlo, il remporte son premier clown de bronze pour la présentation de son groupe mixte de fauves, composé à l'époque de trois lions, deux lionnes et une tigresse. Par la suite, Tom Dieck Junior repart en tournée en Allemagne pour se produire pour le Cirque Busch-Roland, lors de leur tournée \textit{The color of life} en 2008, une tournée durant laquelle on a même pu le voir figurer sur les affiches. Au cours de sa carrière, il se produira également plusieurs fois pour le Cirque Herman Renz en 2007, 2010 et 2012. Durant la saison 2008, Tom Dieck Jr présente son savoir-faire en Russie pour le Großer Russischer Staatscircus, le Grand Cirque d'État de Moscou. À la même époque, il se produira aussi pour la première fois pour le Weltweihnachtscircus, le Cirque de Noël mondial à Stuttgart. Il se présente également à la grande fête lilloise du cirque et au festival international du cirque de Grenoble. En 2009, il se produit au Cirque d'Hiver Bouglione pour leur création qui se nommait \textit{Festif} et fini la saison pour le Fövarosi Nagycirkusz, le Grand Cirque de Budapest en Hongrie. Il profitera de son passage dans les pays de l'Est pour participer au festival du cirque à Varsovie qui lui décernera un clown d'argent. L'hiver suivant, il se retrouva ensuite à un gala de Noël, mais pas n'importe quel gala, il se présenta pour le Grand Cirque de Noël de la famille Bouglione au Bourget. Deux ans plus tard, il remporte une Piste d'Or à l'occasion de la 19\ieme~édition du festival international du cirque de Massy.

Par la suite, il élaborera un nouveau numéro qui sera un tournant important dans sa carrière et le fera entrer dans l'histoire du cirque. En 2012, pour les fêtes, il présentait un tout nouveau numéro composé de cinq tigres, de deux lions blancs et de deux ligrons, le croisement d’un lion et d’une tigresse, à l'occasion du gala du Cirque d'Hiver Bouglione : \textit{Tous à Rio}. L'année suivante, il présente son nouveau numéro mixte pour \textit{Symphonik} la nouvelle création de la maison Arlette Grüss. En 2017, il fait partie de la tournée \textit{Surprise} du Cirque d'Hiver Bouglione. Le 13 janvier 2019, Tom Dieck Junior remporte une Piste d’Argent lors du 27\ieme~festival du cirque de Massy. Au moment de la remise des prix, il annonça prendre sa retraite des pistes.

Tom Dieck Junior est sans nul doute un dresseur qui manque à tout l'univers du cirque. Acclamé par le public pour ses prouesses et salué par ses pairs pour son professionnalisme, Tom Dieck Junior était un dresseur moderne qui utilisait la méthode de Carl Hagenbeck que nous reverrons plus tard. C'était un grand dresseur qui a par ailleurs fait partie du Berufsverband der Tierlehrer, l'association professionnelle des dresseurs allemands, preuve de sa passion pour ses grands félins qui lui ont si bien rendu pendant des années.

\section*{\textit{Sacha Krosemann Jr, la nouvelle génération entre en cage}}
\phantomsection
\addcontentsline{toc}{section}{Sacha Krosemann Jr, la nouvelle génération entre en cage}

Sacha Krosemann Junior est né le 30 juin 2004 à Martigues, dans les Bouches-du-Rhône, à quarante kilomètres de Marseille. Il fait partie d'une famille circassienne depuis sept générations, les Krosemann. Sacha Krosemann Junior est connu pour être l'un des plus jeunes dompteurs de France puisqu'il est actuellement âgé de seulement 19 ans. Dans sa famille, on pratique les différents arts de la piste et l'on est dresseur de fauves de père en fils depuis plus de trois générations. Dès l'enfance, Sacha Junior verra donc son grand-père en cage puis y verra son père, sa voie était pour lui toute tracée : il allait devenir dresseur de fauves à son tour.

C'est donc en voyant son père en cage que Sacha Junior voulut faire dresseur de fauves et par la même occasion perpétuer la tradition familiale. Dès l'âge de 12 ans, il est passionné par les fauves, à tel point qu'il élève une petite tigresse blanche au biberon, ce qui lui a permis d'avoir une réelle complicité avec les fauves.

Sacha Junior apprendra à dresser avec son père qui lui transmettra tout le savoir-faire de leur famille. Il apprendra donc à travailler avec douceur, amour et avec beaucoup de récompenses. Il faudra deux ans à Sacha Junior pour monter son propre numéro de fauve composé de deux tigres blancs, d'un tigre golden tabby qui possède la particularité d'avoir une coloration exceptionnelle due à un allèle récessif et d'un tigre snow. Son numéro est la parfaite conjugaison du dressage traditionnel et moderne, il le prouve d'ailleurs en étant le plus jeune dresseur à faire marcher un tigre sur les pattes arrière en traversant la piste d'un bout à l'autre, ce qui est un exploit pour son âge.

Sacha Junior travaille actuellement au Cirque Europa, le cirque de sa famille, considéré par de nombreux passionnés comme le meilleur cirque traditionnel de France, l'un des seuls à présenter une aussi grande diversité de numéro ainsi qu'un merveilleux groupe de fauves. Sacha Junior partage la tête d'affiche avec son frère Lenny Flavio, avec lequel il présente un numéro de roue de la mort à couper le souffle.

Pour l'avoir vu, je peux dire que j'aime beaucoup le style de Sacha Junior. Je ne suis pas le seul à faire cette analyse, de nombreux passionnés et spécialistes trouvent également son numéro très impressionnant tout en sachant que c'est un numéro qui est encore jeune. À titre personnel, Sacha Junior me fait penser aux jeunes dresseurs de ma jeunesse comme Steeve Caplot. Au début des années 2000, Steeve était un dresseur prometteur et avec du travail et du talent, quelques années plus tard, il fut récompensé dans les festivals de Massy et de Bayeux, le faisant entrer dans l'histoire. Je souhaite à Sacha Jr, non pas d'avoir le même parcours, mais d'être reconnu pour son travail et son savoir-faire. 

\section*{\textit{Théo Leroy, un rêve devenu réalité }}
\phantomsection
\addcontentsline{toc}{section}{Théo Leroy, un rêve devenu réalité}

L'histoire de Théo Leroy commence lorsqu’il est âgé de 4 ans et qu'il fut accompagné par sa tante et sa marraine qui l'emmenèrent au Cirque Arlette Grüss pour le passage annuel du cirque à Arras au mois de mars. C'est de cette manière que Théo découvrit le merveilleux univers du cirque et sut ce qu'il voudrait faire plus tard. En grandissant, Théo gardera toujours son rêve en vue et fera tout pour se donner les moyens afin de transformer son rêve en réalité. La passion du cirque lui permit également de faire de belles rencontres, notamment au Cirque Cilio Ritz qui passait par Le Crotoy, une ville des Hauts-de-France. C'est comme ça qu'à l'âge de seulement 13 ans, Théo décide de partir en tournée avec le Cirque Cilio Ritz pendant cinq ans jusqu'à l'âge de ses 17 ans.

En 2018, il monte avec son ami David Marty une petite entreprise de cirque qu'ils nomment ensemble le Cirque Leroy-Marty. Même si aujourd'hui cette entreprise n'est plus en activité, j'aimerais quand même saluer le génie et le courage qu'ont eu ces jeunes hommes. Théo et David possédaient un barnum rectangulaire pour faire leurs spectacles. Ils se produisaient ainsi comme de vrais banquistes, de ville en ville. Lorsqu'ils tournaient avec leur cirque, ils étaient tous deux trop jeunes pour détenir un permis de conduire et par conséquent, ils transportaient tout leur matériel par vélo. Jamais les contraintes ne découragent les hommes de cirque et cet adage n'échappe pas au duo Leroy-Marty. Aujourd'hui, le seul vestige de cette entreprise de cirque reste leur page Facebook Cirque Leroy-Marty, que je vous conseille de consulter.

Cette aventure fut très enrichissante pour Théo, cependant le vrai monde du cirque finit par lui manquer. Celui de l'odeur de la sciure mélangée à celle du pop-corn qui crépite et de la toile. Ce lieu d'une indicible beauté où se mêle le rêve à l'éphémère et où fleurissent les strass, les paillettes et les moulures rococos. Il se lance donc au Cirque Nicolas Zavatta dirigé par la famille Douchet. À titre informatif, c'est l'un des plus beaux Cirques Zavatta que j'ai pu voir à l’époque où il se nommait encore Cirque Sébastien Zavatta. Aujourd'hui, les deux établissements sont distincts.

Théo arrive au cirque de la famille Douchet en tant que garçon de piste, mais avec l'intention de succéder à Fredo Douchet, monsieur Loyal et directeur du cirque. C'est de cette manière qu'au printemps 2019, Théo Leroy présente les spectacles du Cirque Nicolas Zavatta de manière intermittente. Quelques mois plus tard, en juillet, il en devient officiellement le monsieur Loyal et présente désormais chaque spectacle. Un an plus tard, la famille Douchet décide de lui faire confiance, il présente alors \textit{Évolution} un spectacle à l'époque encore en cours d'écriture. En 2021, Théo met en œuvre et présente un nouveau spectacle qui se nomme \textit{Authentique}. Avec Yann Rossi, un grand clown blanc français, il a réussi à présenter au public un spectacle ancré dans le rêve. En 2022, il effectue sa dernière tournée au Cirque Nicolas Zavatta avec leur création aux inspirations hispaniques, \textit{Olé}.

À partir du 21 janvier 2023, il présente les spectacles du Cirque Royal où il fera sa première représentation à Roubaix. Théo part ensuite en tournée avec le Cirque Royal jusqu'au 16 avril 2023. À la fin de son contrat, personne ne savait où irait Théo. C'est à cette époque que je prends contact avec lui pour parler de son histoire. Il décide de m'annoncer en avant-première qu'à partir du 17 avril 2023, il tournera avec le Nouveau Cirque Zavatta de la famille Falck. À l’heure actuelle, il travaille au cirque Starlight de Tony Production.

\section*{\textit{Arthur et Carmen Möller, un couple de dresseur  }}
\phantomsection
\addcontentsline{toc}{section}{Arthur et Carmen Möller, un couple de dresseur}

L’histoire d’Arthur et Carmen Möller commence en 1960 en Allemagne. Le lieu de rencontre d’Arthur et Carmen Möller a son importance puisqu’ils se sont rencontrés au Cirque Hagenbeck, un des plus grands établissements circassiens d’Allemagne de l’époque. Lorsqu’ils se sont rencontrés, Arthur était dresseur d’éléphant et Carmen, quant à elle, était dresseuse de tigre. À l’époque de leur rencontre, Carmen avait besoin d’une personne qui l’assisterait aux abords de la cage. Arthur accepta et les deux dresseurs purent travailler ensemble. Par la suite, l’amour se mit entre eux et ils se marièrent.

Ensemble, ils auront un enfant qu’ils appelleront Mario Möller. Mario Möller eut une enfance exceptionnelle puisqu’il grandit parmi de nombreux animaux tels que des ours, des lions, des tigres ou encore des serpents.

Grâce à Elie Klant, leur manager, ils ont pu avoir une carrière à la hauteur de leur talent et ont pu présenter un des plus grands spectacles d’ours d’Europe. Cependant, toutes les bonnes choses ont une fin, même au cirque. Lorsque Arthur et Carmen s’apprêtent à accueillir la naissance de leur deuxième fils, Elia Möller, ils décident de prendre leur retraite des pistes et s’installent au zoo d'Hanovre.

Par la suite, Carmen et Arthur Möller décèdent tous deux dans les années 1990. Aujourd'hui, Mario Möller travaille en tant que sellier avec son fils Marcel, ils sont toujours en contact avec les grands cirques allemands comme Roncalli, Krone et le Cirque Belly.

\section*{\textit{Maeven Prein, un artiste complet}}
\phantomsection
\addcontentsline{toc}{section}{Maeven Prein, un artiste complet}

Maeven Prein est né le 24 avril 2001 au Cirque Zavatta Prein. Le Cirque Zavatta Prein est l’une des nombreuses enseignes Zavatta que l’on puisse voir en France, cependant elle est sous la direction d’une grande famille circassienne : la famille Prein. Maeven Prein est donc né au sein d’une famille qui fait du cirque depuis de nombreuses générations.

La première fois que Maeven s'est produit sur une piste de cirque, il n’était que seulement âgé de 9 ans. À l’époque, il commença en tant que jongleur, mais au fil de sa carrière, il présenta en réalité cinq numéros différents. Par la suite, il se spécialisa dans le numéro qui mettra en lumière son cirque : la roue de la mort. Accompagné de son frère, il forme le duo Prein Brothers.

Malgré leur jeune âge et la jeunesse de leur numéro, le public reste captivé par cette performance sur le fil du rasoir. À l’hiver 2023, ils se produisirent sur la mythique pelouse de Reuilly sous le plus grand chapiteau rond du monde, celui du Cirque Mondial de Maxime Kerboua. Placé en numéro final, à chaque représentation, le public hurle, en redemande. Une chose est sûre : l’avenir s’annonce radieux pour les Prein Brothers et pour Maeven Prein.

\section*{\textit{Frédéric Edelstein, l’homme aux douze lions blancs}}
\phantomsection
\addcontentsline{toc}{section}{Frédéric Edelstein, l’homme aux douze lions blancs}

Frédéric Edelstein est né le 30 juillet 1969 à Lyon et il est le fils de Gilbert Edelstein, à l’époque commercial et futur propriétaire du Cirque Pinder. En 1983, alors qu'il n'a que 14 ans, Frédéric voit son père devenir propriétaire du Cirque Pinder. Frédéric se ravit de cette nouvelle vie qui s’offre à lui et très vite se passionne pour le cirque, les animaux et plus particulièrement pour les fauves.

Loin de faire l’unanimité, son père ne voit pas d’un bon œil la nouvelle passion de son fils, toujours à flâner avec les dresseurs près des cages. La carrière de Frédéric Edelstein est le fruit de sa passion développée depuis des années, de son courage, mais également d'un merveilleux coup du sort qui changera sa vie entière.

Un jour, alors que son père est en déplacement, il se dispute avec le dresseur engagé du cirque. Le belluaire décide de ne plus présenter son numéro au spectacle et le Cirque Pinder se retrouve alors sans numéro de fauves. C’est à ce moment que Frédéric sent la chance tourner pour lui. Même s'il n’a tout juste que 14 ans, il décide de faire monter la cage dans l’après-midi, de répéter le numéro que le dresseur présentait habituellement, il le connaissait par cœur avec le temps. C’est ainsi que le soir même, il décida de présenter en cachette de son père un numéro composé de sept tigres.

En l’apprenant, son père furieux lui aurait dit : « je ne t’ai pas mis au monde pour que tu te fasses bouffer par un tigre ! » Malgré la colère de son père, Frédéric sait désormais qu’il veut faire dresseur et rien d’autre. Il décide alors à quelques mois du baccalauréat de le rater et devient dresseur sous la tutelle de Wolfgang Holzmaïr et Dicky Chipperfield, deux grands dresseurs.

Avec le temps, il deviendra la tête d’affiche du cirque Pinder dont il deviendra également directeur. Il faut dire que la volonté de Gilbert était de faire de son fils le Gunter Gebel-Williams français. Même style de costume, même starification, radio et télévision sont devenues également le quotidien de Frédéric. La différence notable entre Gunter et Frédéric est que le premier est dompteur et le second dresseur, mais nous reviendrons à cette distinction plus tard dans cet ouvrage.

Au cours de sa carrière, Frédéric se rendra célèbre avec un numéro mixte, composé de lions, de lionnes, de tigres et de tigresses, pouvant aller de seize à vingt fauves. Dans la seconde partie de sa carrière, il présenta un numéro composé de 12 lions blancs, qu'il travailla avec son maitre Dicky Chipperfield. Hélas, le Cirque Pinder est placé en liquidation judiciaire en mai 2018 et Frédéric se retrouve sans cirque.

En décembre 2018, il est annoncé au programme du Grand Cirque de Noël Américain de Jean Arnaud, mais sa venue sera annulée pour des raisons techniques, c’est Teddy Seneca qui présentera son groupe de lions. Le 24 décembre de la même année, Frédéric se produit à Nantes au Grand Cirque de Noël Medrano. L’année suivante, il fera quelques dates avec le Cirque Claudio Zavatta de la famille Prein qui présentera deux numéros de fauves à chaque représentation : celui de Didier Prein et celui de Frédéric. Depuis quelque temps, Frédéric a cessé de se produire et vit à Pinderland avec ses fauves, dans l’espoir de repartir un jour sur les routes de France.

Symbole d’une génération, Frédéric fut « l’idole des jeunes passionnés », le symbole de l’enfance pour certains, dont je fais partie. Avec ses interventions dans les médias, sa passion et son professionnalisme, Frédéric a su conquérir le cœur des passionnées de cirque. 

\section*{\texorpdfstring{\textit{Roman \& Laurent, un jeune ventriloque prometteur}}{Roman et Laurent, un jeune ventriloque prometteur}}
\phantomsection
\addcontentsline{toc}{section}{Roman \& Laurent, un jeune ventriloque prometteur}

Roman est né le 17 octobre 2009 à Caen, dans une famille qui ne le prédestinait pas au cirque. Très tôt dans son enfance, il se passionna pour le monde du cirque et voulut en faire sa vie. La carrière de Roman commence lorsqu’il fait une rencontre particulière.

Alors en séjour à Londres, il arpente avec ses parents les rayons d’Hamleys, l’un des plus grands magasins de jouets du monde, lorsqu’il vit une marionnette de perroquet. Elle lui fit de la peine et Roman décida de la prendre. Les parents de Roman ont vite compris que sa voie avait été trouvée en le voyant avec.

Au Noël de l’année suivante, il reçut en cadeau une nouvelle marionnette et cette fois-ci, ce fut un orang-outan. Roman décida de le nommer Laurent, faisant ainsi un jeu de mot simple et efficace : « Laurent outan ». C’est ainsi que le duo Roman \& Laurent fut né. Laurent prendra une place particulière dans la vie de Roman qui est fils unique et le reconnait comme un frère. D’ailleurs, dès le départ, il fut ravi de l’arrivée de ce petit orang-outan, car il trouve que les singes sont très proches des humains et il en joue.

Grand admiratif du travail de Michel Dejeneffe et de sa marionnette Tatayet, lui aussi voulait monter un duo humoristique avec sa marionnette. Cependant, il fut confronté à une difficulté importante : il n’arrivait pas à ventriloquer.

Un jour, il se réveilla avec un mal de gorge très particulier. À vrai dire, il n’avait jamais ressenti cette sensation auparavant. Quelques pas en dehors de son lit et il trébucha en laissant s’échapper un éclat de voix qui ne venait pas de ses cordes vocales, mais de son diaphragme. Pour lui, cette chute fut un déclic, il comprit comment ventriloquer.

Avec détermination, il travailla tous les jours afin de maitriser la ventriloquie et affirme que « la ventriloquie est comme de la sculpture, c’est une pâte que l’on façonne ». Il travailla dans l’ombre pendant quatre ans avant de devenir l’un des plus jeunes ventriloques de France.

À l’été 2022, en vacances avec ses grands-parents à Guérande, il réalise son rêve en se produisant pour la première fois sur la piste d’un cirque, celle du Cirque Nicolas Zavatta de la famille Douchet. Il doit le début de sa carrière à Théo Leroy qui décide de le proposer aux propriétaires du cirque qui acceptèrent. C’est de cette manière qu’il se présenta sous le chapiteau du Cirque Nicolas Zavatta à Guérande. Avant de se produire au cirque, il organisait également de nombreux spectacles de rue improvisés dans le but de gagner en expérience pour le jour venu.

De juillet à août 2023, il se produit sur la piste du Cirque Europa de la famille Krosemann. Il se produisit à Cherbourg, Bayeux et dans d’autres villes du nord de la France. Durant cette période, il se présenta également dans d’autres établissements comme le Cirque Rolph Zavatta de la famille Prein ou encore au Cirque Francesco Corbini de la famille Corbini.

Sa plus grande fierté est de voir, lors de son numéro, l’émerveillement du public et le sourire des enfants, non pas pour lui et son talent, mais pour Laurent, sa marionnette, sur laquelle tous les yeux sont rivés.

\section*{\textit{Natalya Jigalova, l’étoile du trapèze ballant }}
\phantomsection
\addcontentsline{toc}{section}{Natalya Jigalova, l’étoile du trapèze ballant}

Natalya Borisnova Vul, plus connue sous le nom de Natalya Jigalova, est née le 21 juillet 1970 à Chișinău, capitale de la Moldavie. La carrière de Natalya Jigalova débute lorsqu’en 1985, elle postula et fut admise à l'école de cirque d'État de Moscou. Dans cette prestigieuse école, elle se forma aux différents arts de la piste et y rencontra son futur mari Andrey Jigalov, futur célèbre clown. 

Dans le cadre de la préparation de son diplôme, avec l’aide de Victor Formine, elle élabore un numéro de trapèze ballant. Un numéro novateur dont seulement quelques personnes avaient le secret et surtout la technique.

En 1989, Natalya a 19 ans, est fraîchement diplômée de son école et est prête à conquérir les plus grandes pistes d’Europe. Enfin, en principe, car en réalité, à la fin de ses études, elle se maria dans la foulée avec Andrey Jigalov et tomba enceinte, ce qui repoussa le début de sa carrière d'artiste. 

Ce qui rendait son numéro de trapèze ballant particulièrement intéressant est qu'elle avait imaginé un système de poulie de manière que la hauteur de son trapèze pourrait être variable. Grâce à cette innovation, elle put commencer son numéro au sol, y inclure de la danse pour ensuite évoluer dans les airs, le tout sans aucune longe de sécurité. Cette différence permit à Natalya Jigalova de se distinguer des autres trapézistes qui devaient monter au trapèze avant le début de leurs numéros.

Le travail de Natalya Jigalova est récompensé une première fois en 1996, lorsqu’elle remporte une médaille d'argent au festival mondial du cirque de demain qui s’est déroulé au cirque d'hiver Bouglione, à Paris. Remporter un prix dans un festival est pour un artiste de cirque la garantie de décrocher des contrats dans les établissements les plus prestigieux du monde. L'enjeu est donc de taille. Avec cette récompense, sa carrière décolla.

Par la suite, elle se produit au Cirque Knie, le cirque national suisse ; au Cirque Roncalli, en Allemagne ; et au Österreichische National-Circus Louis Knie, le Cirque National d’Autriche Louis Knie. Mais également dans des théâtres de variétés, comme au Palais Royal de Kirrwiller. La chance commence à sourire à Natalya Jigalova et en 2003, elle présente son numéro sur la piste du plus célèbre festival de cirque du monde, celle du festival international du cirque Monte-Carlo. À cette occasion, on lui décerne le prix du Cirque de Budapest. Si tout sourit à Natalya Jigalova sur le plan professionnel, sur le plan personnel, c'est un peu plus complexe puisqu’entre-temps, elle se sépare de son mari Andrey Jigalov.

Désormais mère célibataire et connaissant la précarité de la vie d’artiste, elle décide de prendre sa retraite des pistes et de reprendre des études. Elle obtient alors un diplôme de psychothérapeute, qui ne lui servira que très peu, puisqu'elle regagnera vite le monde du cirque dans une toute nouvelle fonction.	

Elle décide d’accepter l’offre de Maskim Nikouline et devient régisseuse de piste du Cirque Nikouline, le plus célèbre cirque russe. Elle se plait très vite dans ce poste et ses expériences dans les différents établissements d’Europe en font une régisseuse active, professionnelle, un véritable élément moteur du cirque. De 2016 à 2018, elle est régisseuse de piste du festival du cirque du Val-d'Oise. Hélas, on lui diagnostiqua tardivement un cancer du côlon dont elle ne put guérir. Elle rendit son dernier souffle le 13 juin 2022, à Moscou, à l'âge de 52 ans. Après un hommage reçu au Cirque Nikouline, elle fut enterrée au cimetière Khovansky, à Moscou.

\section*{\textit{Henri Dantès, le dompteur du plus grand cirque du monde }}
\phantomsection
\addcontentsline{toc}{section}{Henri Dantès, le dompteur du plus grand cirque du monde}
Heinrich Honvehlmann, plus connu sous le nom d’Henri Dantès, est né le 17 août 1932 à Datteln en Allemagne dans une famille d’industriels et par conséquent, rien ne le prédisposait à la prestigieuse carrière qu’il aura au cirque. Certains hommes sont arrivés au cirque par passion, d’autres par hasard et d’autres par amour pour une femme.

Alors que le Cirque Bouglione avait planté son chapiteau à Munich, Henri Dantès y rencontre une trapéziste, dont il tombe éperdument amoureux et décide de la suivre. C’est ainsi qu’en 1952, Henri Dantès part en tournée avec le Cirque Bouglione.

Si à ses débuts, il fut le garçon de cage de Firmin Bouglione, très vite, Firmin décela en lui un potentiel rare en voyant la passion qu’il avait pour les fauves. C’est ainsi qu'il le prit comme élève et lui apprit le noble art de la dompte. Il existe une anecdote plutôt cocasse concernant le début de sa carrière et c’est Henri lui-même qui la raconta dans un documentaire de 1992 réalisé par Eric Sandrin, aujourd’hui malheureusement introuvable. Devant la caméra, il anéantit le mythe du dresseur sans peur et avoue qu’au début de sa carrière, il était tétanisé par la peur à l’idée d’entrer en cage. Il avoue même avoir quelquefois pleuré.

Autre anecdote, l’origine de son nom de piste. Heinrich Honvehlmann a décidé de s’appeler Henri Dantès pour plusieurs raisons. Henri est une francisation de son prénom et Dantès fait référence à Edmond Dantès, le héros de la célèbre œuvre d'Alexandre Dumas : \textit{Le Comte de Monte-Cristo}.

Durant sa carrière, il se spécialisa dans les animaux sauvages. Il présenta donc plusieurs groupes de fauves avec des tigres, des lions, des panthères, mais également des ours. Sa grande spécialité reste cependant les tigres et les lions.

L’un des numéros qui le rendit célèbre était composé d’un groupe de lions mâles. Pour conclure son numéro, il effectuait un exercice particulièrement dangereux dans lequel chaque lion venait tour à tour s’allonger sur lui.

Son travail de qualité lui fit une réputation dans le monde du cirque. À l'époque, il eut l'opportunité de travailler dans les plus illustres cirques français comme au Cirque Pinder, au Cirque Amar, au Cirque Grüss ou encore au Cirque Jean Richard.

La carrière d’Henri Dantès sera également marquée par des tournages dans différents films. En 1956, il joue la doublure de Burt Lancaster dans le film \textit{Trapèze} de Carol Reed. En 1964, il interprète le rôle d’Emile Schuman, un dresseur de fauves terrorisé à l’idée de rentrer en cage avec des tigres dans \textit{Le Plus Grand Cirque du Monde} d'Henry Hathaway. En 1966, il tourne un dernier film loin des sentiers dorés du cirque en jouant dans \textit{La Bible} de John Huston.

Henri Dantès fit également plusieurs apparitions dans l’émission La Piste aux étoiles de Gilles Margaritis. En 1967, il y présentera notamment un groupe de tigres au Cirque d’Hiver Bouglione, dont les images sont encore visionnables aujourd’hui. En 1972, à l’occasion du 39\ieme~ gala des artistes présenté par Jerry Lewis au Cirque d’Hiver Bouglione, Jean-Claude Brialy devient disciple d'Henri Dantès qui le forme au métier de dresseur. Jean-Claude Brialy présenta ainsi, le temps d'une soirée, le groupe de fauves d’Henri Dantès.

À la fin de sa carrière, il tourna dans de petits établissements comme le Cirque Roger Lanzac dans les années 1990. Il travailla également dans des zoos et s’efforcera de transmettre son savoir-faire acquis durant toutes ses années à travailler avec des fauves. Henri Dantès s’éteint le 28 février 1997 à Bordeaux à l’âge de 64 ans.

\section*{\textit{Michel Palmer, Monsieur Loyal des grands cirques }}
\phantomsection
\addcontentsline{toc}{section}{Michel Palmer, Monsieur Loyal des grands cirques}

f
\section*{\textit{Roger Falck, la fierté française à Monte-Carlo }}
\phantomsection
\addcontentsline{toc}{section}{Roger Falck, la fierté française à Monte-Carlo}

f




\thispagestyle{empty} %Dernière page vide