\chapter{Les visages du cirque traditionnel}
\section*{\textit{Even Landri, le gladiateur des temps modernes}}
\phantomsection
\addcontentsline{toc}{section}{Even Landri, le gladiateur des temps modernes}

Even Landri est né le 2 août 1975 à Toulon dans une famille d'origine italienne et circassienne depuis plusieurs générations. La famille Landri est originaire de Torre Annunziata, une station balnéaire et thermale italienne à une vingtaine de kilomètres de Naples. Dès son enfance, Even vivait dans une famille de cirque traditionnel qui pratiquait les différents arts de la piste et possédait également de nombreux animaux sauvages. C'est dans cet environnement familial que rapidement, il côtoya de nombreux animaux, des plus communs aux plus atypiques. C'est dans cette enfance si particulière qu'il fut bercé par le rugissement des fauves, dont il se passionna très rapidement alors qu'il n'avait que trois ans.

Il faut savoir que dans les familles circassiennes traditionnelles, les enfants s'initient souvent aux différents arts de la piste et présentent différents numéros au cours de leur vie. Durant leur jeunesse, ils prennent souvent plusieurs années pour trouver leur voie, le numéro qu'il leur plait particulièrement ou alors un numéro dans lequel ils possèdent des facilités, leur spécialité. Trouver sa spécialité n'est donc pas simple face à la multitude de numéros possibles au cirque. Cependant, le cas des dresseurs de fauves reste assez particulier puisque les enfants qui se passionnent pour les fauves et entrent en cage dès l'enfance en font le plus souvent leur voie, c'est le cas d'Even Landri. 

C'est grâce à son oncle Jean Landri que le jeune Even âgé de seulement 16 ans fit sa première entrée en cage en sa compagnie et qui par la suite lui apprit tout l'art du dressage. C'est en répétant sous l'œil éclairé de son oncle qu’en 1992, à l'âge de 17 ans, il présenta son premier numéro composé à l'époque de trois tigres, numéro auquel il intégra quatre autres tigres l'année suivante. Treize ans plus tard, en 2006, Even décida de monter un numéro mixte de fauves, c’est-à-dire composé de différentes espèces. Ce numéro composé de quatre tigres et de quatre lionnes, il le présenta en public seulement un an après son élaboration, une prouesse quand on sait qu'un numéro animalier prend habituellement plusieurs années à être monté. Durant mon enfance, c'est grâce à ce numéro que j'ai connu, l'homme que l'on surnomme encore aujourd'hui \textit{Le gladiateur des temps modernes}. Quatre ans plus tard, en 2010, Even Landri intègra deux tigres blancs à son numéro mixte de fauves, désormais composé de quatre lionnes et de six tigres. Il est d'ailleurs important de noter que les tigres blancs ne représentent pas une race de tigre à part entière, ils représentent le plus souvent une variété de tigres du Bengale. 

Sa méthode pour dresser ses fauves commence dès leur naissance avec de l'éducation à l'âge de quatre mois avec la mère. Par la suite, il commencera ce que l'on peut considérer comme les bases du dressage à 10 mois. Comme la plupart des dresseurs, il travaille à la récompense, c'est-à-dire en morceaux de viande. Cette technique, bien connue du milieu du dressage, permet un apprentissage dans le jeu et le respect de l'animal. Even Landri est le seul dresseur de France à faire marcher un tigre sur ces pattes arrière sans chambrière ni bambou, ces deux instruments ne servant uniquement à donner des indications aux fauves sans jamais les toucher.

Actuellement, il finit chacun de ses numéros les mains nues, ce qui est exceptionnel dans le milieu du dressage de fauves et surtout extrêmement dangereux. Il travaille au Cirque de Venise, le cirque de sa famille dans lequel il présente tous les animaux du cirque excepté la cavalerie présentée par Steeve Landri, son frère. À l'heure actuelle, il monte un nouveau numéro mixte de fauves avec un tigre snow, une variété de tigres blancs qui ont la particularité d'avoir des rayures si claires que l'on pourrait croire qu'ils n'en possèdent pas.

J'ai connu Even Landri grâce à une vidéo de son numéro qui était sur internet. Il existe de nombreuses captations des numéros de fauves d'Even Landri, chacun de ses numéros sont d'ailleurs exceptionnels, c'est pourquoi Even Landri est considéré par ses pairs comme l'un des dresseurs de fauves les plus talentueux de l'époque. La vidéo qui me l'a fait découvrir a été tournée par Jon Notenboom le 25 octobre 2009 à Carpentras avec l'accord du dresseur. Cette vidéo est encore disponible 13 ans après sa mise en ligne. C'est l'un de mes dresseurs préférés, il est le parfait représentant d'un dressage traditionnel créatif, un dressage à l'italienne dont se dégage un panache éclatant. Je vous invite donc à voir ou revoir cette vidéo pour vous faire votre avis sur celui que l’on nomme encore aujourd'hui, et de manière justifiée, \textit{Le gladiateur des temps modernes}.

\section*{\textit{Pierre Marchand, le virtuose du diabolo}}
\phantomsection
\addcontentsline{toc}{section}{Pierre Marchand, le virtuose du diabolo}

Pierre Mazieri, plus connu sous son nom de scène « Pierre Marchand », est né un 5 septembre à Saint-Mandé. Il est l'ainé d'une fratrie de trois enfants d'origine Corse. Pierre Marchand n'est pas né d'une famille circassienne, ses parents sont tous deux enseignants, rien ne le prédestinait donc à devenir l'artiste talentueux qu'il allait devenir. Bien qu'il soit né en France, il vécut une partie de son enfance au Togo grâce au déplacement professionnel de ses parents. En Afrique, il connut une enfance heureuse et fut un enfant plein d'énergie. Malheureusement, quelque temps plus tard, à l'âge de 7 ans, ses parents durent rentrer en France. Pierre ressentit comme un déchirement l'abandon de sa terre presque natale.

Peu habitué à son nouvel environnement, l'Île-de-France, il supporta mal le rythme de la ville et déborda vite d'énergie. Au fil du temps, ses parents peinèrent à le canaliser et en septembre, sa mère décida de l'inscrire à l'école nationale du cirque à Paris, dirigée par Annie Fratellini et Pierre Etaix. Aujourd'hui cette école n'existe plus, mais elle a été succédée par l'académie Annie Fratellini. C'est grâce à cette inscription que tous les mercredis et samedis après-midi Pierre se défoula tout en apprenant également de nouvelles disciplines. Pierre se plut tellement dans ce nouvel univers qu'est le cirque et dans cette école qu'en décembre, il fut remarqué par Annie Fratellini qui décela en lui une future âme d'artiste. Elle décida alors de le prendre sous son aile afin de lui transmettre son savoir-faire : les bases du jonglage.

Ses après-midi, il les passa à l'école du cirque avec son maitre Italo Medini, un célèbre jongleur qui lui apprit toutes les subtilités du diabolo. À partir de cet instant, Pierre eut une scolarité à horaires aménagés dans laquelle il put pleinement s'épanouir, le parfait mélange de l'école et du cirque. Pendant 9 ans, il tint ce rythme de trente heures de diabolo par semaine jusqu'à l'obtention de son bac scientifique. Par la suite, il décida de lancer sa carrière avec son numéro qu’il venait de finir d'élaborer.

Son numéro qu'il présente encore aujourd'hui et le même depuis le début de sa carrière même s'il a dû monter des numéros plus condensés pour l'émission \textit{La France à un incroyable talent}. Pendant les sept minutes de numéro qu'il propose, ce soleil corse nous donne de l'énergie pure à tel point qu'à chaque numéro, il perd 1,4 kg. C'est à cette époque qu'il décida de lancer sa carrière de jongleur professionnel, et cette carrière fut propulsée par une rencontre en particulier, celle de Vincent Lagaff.

Vincent Lagaff lui proposa de présenter son numéro dans son émission \textit{Le Bigdil}, et par la même occasion lança la carrière de celui qui se fit connaître mondialement comme Pierre Marchand. C'est aussi à cette époque qu'il se produisit dans de nombreux festivals de cirque à travers le monde. En 2004, il remporta par exemple la médaille d'or au festival du cirque de Wiesbaden en Allemagne et en 2006 au festival du cirque de Budapest, le prix du Cirque de Moscou et le prix de la ville de Budapest. Lors de la 31\ieme~édition du festival international du cirque de Monte-Carlo, il reçut une standing ovation du public et remporta le prix du Club du Cirque. La même année, en 2007, il remporta un Loyal d'or, la plus haute récompense décernée à l'occasion de la 2\ieme~édition du festival international du cirque de Bayeux.

Grâce à ses nombreuses récompenses, il parvint à travailler dans les établissements de spectacles les plus prestigieux du monde. À partir de 2006, il signa un contrat au Lido, l'un des plus célèbres cabarets parisiens, où il se produisit pendant 8 ans pour la revue \textit{Bonheur}, avant de rejoindre le Cirque d'Hiver Bouglione pour la tournée événement \textit{Bravo}. Il se produisit également deux fois dans l'émission de Patrick Sébastien, \textit{Le Plus Grand Cabaret du monde}, notamment en 2015.

En 2012, qu'il participa à l'émission de télévision \textit{La France a un incroyable talent} qui le fit connaitre au grand public. L'un des candidats préférés du public, il arriva en première place en demi-finale, mais hélas termina la finale en cinquième place. En 2016, il fit un remplacement de quelques jours au Moulin Rouge avant de partir l'année suivante pour le Danemark et la Suède. En 2018, il rentra en France pour répondre aux appels de Pierre Meyer, qui le sollicitait depuis des années, et se présenta au Royal Palace Kirrwiller. En 2019, après une brève escale aux Etats-Unis, il se produisit pour l'Europa Park, le plus grand parc d'attraction d'Allemagne, puis il repartit pour le Brésil, pour la Suisse. 

Évidemment, au cours de sa carrière, en plus de se produire dans des cabarets prestigieux, il a également participé à des spectacles dans des cirques renommés à travers le monde. Parmi eux, on compte les Cirques Krone, Roncalli et Flic Flac en Allemagne ; le Cirque national norvégien ; le Cirque Tihany au Mexique, ainsi que la tournée \textit{Excentrik} du Cirque Arlette Grüss en 2021. Les 4 et 5 décembre 2021, il a également pris part au spectacle \textit{La H Arena fait son cirque} à Nantes, sous la direction artistique de Joseph Bouglione. Récemment, il a été annoncé au programme anniversaire des 40 ans du Cirque Arlette Grüss qui débutera en octobre 2024. 

En tant qu'artiste jongleur, Pierre Marchand a toujours eu une place particulière dans mon cœur, puisque c'est lui qui me donna envie de faire du diabolo ma spécialité. Pour moi comme pour de nombreux passionnés, même si ce n’est pas le plus technique des jongleurs, il a toujours été au-dessus d'une certaine manière des autres diabolistes. Pendant des heures, je me rappelle encore apprendre tous ses gestes, ses mimiques faciales, son style si particulier qu'encore aujourd'hui, en 2024, je retrouve malgré moi dans ma manière de jongler. Il y aura toujours une part de Pierre Marchand en moi.

\section*{\textit{Martin Lacey Junior, l'homme au clown d'or}}
\phantomsection
\addcontentsline{toc}{section}{Martin Lacey Junior, l'homme au clown d'or}

Martin Lacey Junior est né le 8 juin 1977 à Sunderland au Royaume-Uni, dans une fratrie de trois enfants. Ses parents étaient directeurs de zoo et dresseurs d’animaux. En effet, durant de nombreuses années, son père Martin Lacey Senior, également directeur du Big Apple Circus, travailla avec un groupe de lions tandis que sa mère Susan travailla avec un groupe mixte de tigres et de lions. Susan Lacey reçut d'ailleurs un clown d'argent en 2005 pour la présentation de son groupe de tigres blancs, tandis que son fils, Alexander, le frère aîné de Martin, reçut un clown d'argent pour la présentation de son groupe mixte en 2003. On peut donc voir que la spécialité de la famille Lacey est le dressage de fauves. Dès la naissance, Martin vécut dans cet environnement dans lequel il fut au plus proche de nombreux animaux. Très vite, il se passionna pour les fauves avec lesquelles il rêva secrètement de travailler un jour.

Comme de nombreux enfants circassiens, il suivit dans un premier temps une éducation itinérante, mais en 1988, les parents du jeune Martin, âgé à l’époque de 11 ans, furent de plus en plus soucieux de lui inculquer un bon niveau d'instruction et décidèrent de l'inscrire en pensionnat. Il fit donc une partie de sa scolarité à la Cordeaux Highschool dans le comté du Lincolnshire, dans la région des Midlands de l'Est en Angleterre. C'est durant cette période de sa scolarité qu'il se passionna pour de nouvelles disciplines comme le rugby et la boxe.

En 1994, Martin eut désormais 17 ans et après avoir obtenu son baccalauréat, il ne pensa qu'à une seule chose : revenir au cirque, retrouver sa famille et ses animaux. Cependant, son père décida alors de mettre ses études à profit et le délégua au service marketing. Martin aimait le domaine de la publicité dans lequel il se plaisait, mais rapidement l'appel des fauves devint trop fort, et il choisit de se lancer en tant que dresseur de fauves, marchant ainsi dans les pas de la tradition familiale.

Il rejoint donc son frère Alexander et monte son premier numéro mixte de fauves. Martin Lacey Junior fit alors ses débuts au Cirque Pauwels et au Cirque Kino's, avant de rejoindre le Cirque Krone. Trois ans plus tard, Martin, désormais âgé de 20 ans, travailla seul son numéro de lion avec lequel il se fit connaitre notamment au Cirque Krone. Il y posa d’ailleurs ses valises après s'être marié avec Jana Madana la fille de Christel Sembach-Krone, la directrice du Cirque Krone, l'un des plus grands cirques d'Allemagne.

En 1999, il se présenta au festival international du cirque de Massy avec son numéro composé à l'époque de lionnes et reçut un chapiteau de cristal. En janvier 2000, il gagna un clown d'argent avec son groupe de lionnes lors de la 20\ieme~édition du festival international du cirque de Monte-Carlo. En 2004, il obtint une étoile d'or au festival international du cirque Auvergne Rhône-Alpes Isère. En 2010, ce fut la consécration pour Martin qui remporta à 33 ans un clown d'or pour son numéro composé de lionnes et d’un lion. En 2019, il reproduisit l'exploit et obtint de nouveau un clown d'or pour son numéro et par la même occasion entra définitivement dans l'histoire. Il est également important de noter qu'en 2016, pour la 40\ieme~édition du festival international du cirque de Monte-Carlo, Martin Lacey Junior partagea la piste avec des légendes du dressage de fauves que sont Nicolaï Pavlenko et Massimiliano Nones.

Grâce à ses nombreuses récompenses qui firent sa carrière, il se produisit à travers le monde, notamment pour le Cirque d'Hiver Bouglione en 2011 lors du gala \textit{La Perle du Bengale} au Bourget. Martin Lacey Junior a pour objectif de prouver aux yeux du monde que le dressage de fauves est un art en organisant régulièrement des entrainements ouverts au public, afin de montrer son savoir faire.

Martin Lacey Junior a longtemps été l'un de mes dresseurs de fauves préférés, aux côtés de Frédéric Edelstein et de Steeve Caplot, c'est l'exemple même du dressage moderne et en douceur. Si vous deviez retenir uniquement deux de ses numéros, je vous conseille de voir ses prestations lors de la 30\ieme~et de la 34\ieme~édition du festival international du cirque de Monte-Carlo.

\section*{\textit{Banbino Mouredon, et le baiser de la mort}}
\phantomsection
\addcontentsline{toc}{section}{Banbino Mouredon, et le baiser de la mort}

Banbino Mouredon est né le 5 septembre 1940 au Grand-Bourg, à une vingtaine de kilomètres de Guéret dans la Creuse. Sa famille, la famille Mouredon, est originaire du Gard dans le sud de la France et circassienne depuis sept générations.

Banbino est connu pour avoir une grande carrière de dresseur de fauve. Il se fit notamment connaitre pour être l'un des rares dresseurs de fauves français de l'époque à présenter le baiser de la mort avec un lion. En parallèle de sa carrière de dresseur, il fut également directeur de son cirque familial et tourna sous différentes enseignes comme le Mondial Circus, le Cirque Annie Fratellini ou encore le fameux Cirque Achille Zavatta

En 2018, à l'âge de 78 ans, il se présenta pour la dernière fois sur une piste de cirque à Neuilly-sur-Marne en Seine-Saint-Denis. Le 5 septembre 2022, il fêta ses 83 ans, toujours avec le sourire et heureux, le vieux lion est désormais à la retraite.

\section*{\textit{Tom Dieck Junior, l'homme qui parlait à l'oreille des ligrons}}
\phantomsection
\addcontentsline{toc}{section}{Tom Dieck Junior, l'homme qui parlait à l'oreille des ligrons}

Tom Dieck Junior est né le 30 avril 1983 à Montbrison, à une quarantaine de kilomètres de Saint-Etienne, dans la Loire. Il est le fils de Tom et de Gilian Dieck, de célèbres dresseurs de fauves, mais aussi le petit-fils de Tom Dieck Senior. Il fait partie de la famille Dieck, une famille circassienne depuis plusieurs générations, c'est donc tout naturellement que le fils voulu faire comme le père, comme le père avait voulu faire comme son père avant lui, afin de perpétuer la tradition familiale.

En décembre 2003, alors qu'il n'avait que 21 ans, Tom Dieck Junior mit en place un numéro mixte de fauves et le présenta au Cirque Jules Verne à Amiens. Cela marqua le début d'une carrière fulgurante qui s'annonçait pour lui. Deux ans plus tard, il rejoignit la maison Arlette Grüss pour leur création \textit{Rêves}, qui fêta le vingtième anniversaire d'existence du cirque, avant de s'exporter l'année suivante en Allemagne pour se présenter au Cirque Probst. C'est à cette époque qu'il remporta le Muermans-Vastgoed Circus Award.

L'année suivante en 2006, il présenta son groupe mixte de fauves lors de la 14\ieme~édition du festival international du cirque de Massy et remporta un chapiteau de cristal. La même année, il gagna le 2\ieme~prix du jury à l'occasion du 11\ieme~festival du cirque d'Enschede aux Pays-Bas. En 2007, à l'occasion de la 31\ieme~édition du festival international du cirque de Monte-Carlo, il reçut son premier clown de bronze pour la présentation de son groupe mixte de fauves, composé à l'époque de trois lions, deux lionnes et une tigresse.

Par la suite, Tom Dieck Junior repartit en tournée en Allemagne afin de se produire pour le Cirque Busch-Roland, lors de leur tournée \textit{The color of life}, en 2008. Une tournée durant laquelle on avait même pu le voir figurer sur les affiches. Au cours de sa carrière, il se produisit également plusieurs fois pour le Cirque Herman Renz en 2007, 2010 et 2012. Durant la saison 2008, Tom Dieck Junior présenta son savoir-faire en Russie pour le Großer Russischer Staatscircus, le Grand Cirque d'État de Moscou. À la même époque, il se produisit aussi pour la première fois pour le Weltweihnachtscircus, le Cirque de Noël de Stuttgart. Il se présenta également à la grande fête lilloise du cirque et au festival international du cirque de Grenoble. En 2009, il se produisit au Cirque d'Hiver Bouglione pour leur création qui se nommait \textit{Festif} et finit la saison au Fövarosi Nagycirkusz, le Grand Cirque de Budapest en Hongrie. Il profita de son passage dans les pays de l'Est pour participer au festival du cirque à Varsovie qui lui décerna un clown d'argent. L'hiver suivant, il se retrouva ensuite à un gala de Noël, mais pas n'importe quel gala, il se présenta pour le Grand Cirque de Noël de la famille Bouglione au Bourget. Deux ans plus tard, il remporta une piste d'or à l'occasion de la 19\ieme~édition du festival international du cirque de Massy.

Par la suite, il élabora un nouveau numéro qui fut un tournant important dans sa carrière et le fit entrer dans l'histoire du cirque. En 2012, pour les fêtes, il présenta un tout nouveau numéro composé de cinq tigres, de deux lions blancs et de deux ligrons, le croisement d’un lion et d’une tigresse, à l'occasion du gala du Cirque d'Hiver Bouglione : \textit{Tous à Rio}. L'année suivante, il présenta son nouveau numéro mixte pour \textit{Symphonik} la nouvelle création de la maison Arlette Grüss. En 2017, il fait partie de la tournée \textit{Surprise} du Cirque d'Hiver Bouglione. Le 13 janvier 2019, Tom Dieck Junior obtint une piste d’argent lors du 27\ieme~festival du cirque de Massy. Au moment de la remise des prix, il annonça prendre sa retraite des pistes.

Tom Dieck Junior est sans nul doute un dresseur qui manque à tout l'univers du cirque. Acclamé par le public pour ses prouesses et salué par ses pairs pour son professionnalisme, Tom Dieck Junior était un dresseur moderne qui utilisait la méthode de Carl Hagenbeck que nous reverrons plus tard. C'était un grand dresseur qui a par ailleurs fait partie du Berufsverband der Tierlehrer, l'association professionnelle des dresseurs allemands, preuve de sa passion pour ses grands félins qui lui ont si bien rendu pendant des années.

\section*{\textit{Sacha Krosemann Junior, la nouvelle génération entre en cage}}
\phantomsection
\addcontentsline{toc}{section}{Sacha Krosemann Junior, la nouvelle génération entre en cage}

Sacha Krosemann Junior est né le 30 juin 2004 à Martigues, dans les Bouches-du-Rhône, à quarante kilomètres de Marseille. Il fait partie d'une famille circassienne depuis sept générations, la famille Krosemann. Sacha Krosemann Junior est connu pour être l'un des plus jeunes dompteurs de France puisqu'il est actuellement âgé de seulement 20 ans. Dans sa famille, on pratique les différents arts de la piste et on est dresseur de fauves de père en fils depuis plus de trois générations. Dès l'enfance, Sacha Junior vit donc son grand-père en cage puis y vit son père, sa voie était pour lui toute tracée : il allait devenir dresseur de fauves à son tour.

C'est donc en voyant son père en cage que Sacha Junior voulut faire dresseur de fauves et par la même occasion perpétuer la tradition familiale. Dès l'âge de 12 ans, il fut passionné par les fauves, à tel point qu'il éleva une petite tigresse blanche au biberon, ce qui lui a permis d'avoir une réelle complicité avec les fauves.

Sacha Junior apprit à dresser avec son père qui lui transmit tout le savoir-faire de leur famille. Il apprit donc à travailler avec douceur, amour et avec beaucoup de récompenses. Il fallut deux ans à Sacha Junior pour monter son propre numéro de fauves composé de deux tigres blancs, d'un tigre golden tabby qui possède la particularité d'avoir une coloration exceptionnelle due à un allèle récessif et d'un tigre snow. Son numéro est la parfaite conjugaison du dressage traditionnel et moderne, il le prouve d'ailleurs en étant le plus jeune dresseur à faire marcher un tigre sur les pattes arrière en traversant la piste d'un bout à l'autre, ce qui est un exploit pour son âge.

Sacha Junior travaille actuellement au Cirque Europa, le cirque de sa famille, considéré par de nombreux passionnés comme le meilleur cirque traditionnel de France, l'un des seuls à présenter une aussi grande diversité de numéro ainsi qu'un merveilleux groupe de fauves. Sacha Junior partage la tête d'affiche avec son frère Lenny Flavio, avec lequel il présente un numéro de roue de la mort à couper le souffle.

Pour l'avoir vu, je peux dire que j'aime beaucoup le style de Sacha Junior. Je ne suis pas le seul à faire cette analyse, de nombreux passionnés et spécialistes trouvent également son numéro très impressionnant tout en sachant que c'est un numéro qui est encore jeune. À titre personnel, Sacha Junior me fait penser aux jeunes dresseurs de ma jeunesse comme Steeve Caplot. Au début des années 2000, Steeve était un dresseur prometteur et avec du travail et du talent, quelques années plus tard, il fut récompensé dans les festivals de Massy et de Bayeux, le faisant entrer dans l'histoire. Je souhaite à Sacha Junior, non pas d'avoir le même parcours, mais d'être reconnu pour son travail et son savoir-faire. 

\section*{\textit{Théo Leroy, un rêve devenu réalité }}
\phantomsection
\addcontentsline{toc}{section}{Théo Leroy, un rêve devenu réalité}

L'histoire de Théo Leroy commença lorsqu’il eut 4 ans et qu'il fut accompagné par sa tante et sa marraine qui l'emmenèrent au Cirque Arlette Grüss lors du passage annuel du cirque à Arras au mois de mars. C'est de cette manière que Théo découvrit le merveilleux univers du cirque et sut ce qu'il voudrait faire plus tard. En grandissant, Théo gardera toujours son rêve en vue et fera tout pour se donner les moyens afin de transformer son rêve en réalité. La passion du cirque lui permit également de faire de belles rencontres, notamment au Cirque Cilio Ritz qui passait par Le Crotoy, une ville des Hauts-de-France. C'est comme ça qu'à l'âge de seulement 13 ans, Théo décide de partir en tournée avec le Cirque Cilio Ritz pendant cinq ans jusqu'à l'âge de ses 17 ans.

En 2018, il monte avec son ami David Marty une petite entreprise de cirque qu'ils nomment ensemble le Cirque Leroy-Marty. Même si aujourd'hui cette entreprise n'est plus en activité, j'aimerais quand même saluer le génie et le courage qu'ont eu ces jeunes hommes. Théo et David possédaient un barnum rectangulaire pour faire leurs spectacles. Ils se produisaient ainsi comme de vrais banquistes, de ville en ville. Lorsqu'ils tournaient avec leur cirque, ils étaient tous deux trop jeunes pour détenir un permis de conduire et par conséquent, ils transportaient tout leur matériel par vélo. Jamais les contraintes ne découragent les hommes de cirque et cet adage n'échappe pas au duo Leroy-Marty. Aujourd'hui, le seul vestige de cette entreprise de cirque reste leur page Facebook Cirque Leroy-Marty, que je vous conseille de consulter.

Cette aventure fut très enrichissante pour Théo, cependant le vrai monde du cirque finit par lui manquer. Celui de l'odeur de la sciure mélangée à celle du pop-corn qui crépite et de la toile. Ce lieu d'une indicible beauté où se mêle le rêve à l'éphémère et où fleurissent les strass, les paillettes et les moulures rococos. Il se lance donc au Cirque Nicolas Zavatta dirigé par la famille Douchet. À titre informatif, c'est l'un des plus beaux Cirques Zavatta que j'ai pu voir à l’époque où il se nommait encore Cirque Sébastien Zavatta. Aujourd'hui, les deux établissements sont distincts.

Théo arrive au cirque de la famille Douchet en tant que garçon de piste, mais avec l'intention de succéder à Fredo Douchet, Monsieur Loyal et directeur du cirque. C'est de cette manière qu'au printemps 2019, Théo Leroy présente les spectacles du Cirque Nicolas Zavatta de manière intermittente. Quelques mois plus tard, en juillet, il en devient officiellement le Monsieur Loyal et présente désormais chaque spectacle. Un an plus tard, la famille Douchet décide de lui faire confiance, il présente alors \textit{Évolution} un spectacle à l'époque encore en cours d'écriture. En 2021, Théo met en œuvre et présente un nouveau spectacle qui se nomme \textit{Authentique}. Avec Yann Rossi, un grand clown blanc français, il a réussi à présenter au public un spectacle ancré dans le rêve. En 2022, il effectue sa dernière tournée au Cirque Nicolas Zavatta avec leur création aux inspirations hispaniques, \textit{Olé}.

À partir du 21 janvier 2023, il présente les spectacles du Cirque Royal où il fera sa première représentation à Roubaix. Théo part ensuite en tournée avec le Cirque Royal jusqu'au 16 avril 2023. À la fin de son contrat, personne ne savait où irait Théo. C'est à cette époque que je prends contact avec lui pour parler de son histoire. Il décide de m'annoncer en avant-première qu'à partir du 17 avril 2023, il tournera avec le Nouveau Cirque Zavatta de la famille Falck. À l’heure actuelle, il travaille au cirque Starlight de Tony Production.

\section*{\textit{Arthur et Carmen Möller, un couple de dresseur  }}
\phantomsection
\addcontentsline{toc}{section}{Arthur et Carmen Möller, un couple de dresseur}

L’histoire d’Arthur et Carmen Möller commence en 1960 en Allemagne. Le lieu de rencontre d’Arthur et Carmen Möller a son importance puisqu’ils se sont rencontrés au Cirque Hagenbeck, un des plus grands établissements circassiens d’Allemagne de l’époque. Lorsqu’ils se sont rencontrés, Arthur était dresseur d’éléphant et Carmen, quant à elle, était dresseuse de tigre. Du temps de leur rencontre, Carmen avait besoin d’une personne qui l’assisterait aux abords de la cage. Arthur accepta et les deux dresseurs purent travailler ensemble. Par la suite, l’amour se mit entre eux et ils se marièrent.

Ensemble, ils auront un enfant qu’ils appelleront Mario. Mario Möller eut une enfance exceptionnelle puisqu’il grandit parmi de nombreux animaux tels que des ours, des lions, des tigres ou encore des serpents.

Grâce à Elie Klant, leur manager, la famille Möller a pu avoir une carrière à la hauteur de son talent et a pu présenter un des plus grands spectacles d’ours d’Europe. Cependant, toutes les bonnes choses ont une fin, même au cirque. Lorsque Arthur et Carmen s’apprêtent à accueillir la naissance de leur deuxième fils, Elia Möller, ils décident de prendre leur retraite des pistes et s’installent au zoo d'Hanovre.

Par la suite, Carmen et Arthur Möller décèdent tous deux dans les années 1990. Aujourd'hui, Mario Möller travaille en tant que sellier avec son fils Marcel, ils sont toujours en contact avec les grands cirques allemands comme Roncalli, Krone et le Cirque Belly.

\section*{\textit{Maeven Prein, un artiste complet}}
\phantomsection
\addcontentsline{toc}{section}{Maeven Prein, un artiste complet}

Maeven Prein est né le 24 avril 2001 au Cirque Zavatta Prein. Le Cirque Zavatta Prein est l’une des nombreuses enseignes Zavatta que l’on puisse voir en France, cependant elle est sous la direction d’une grande famille circassienne : la famille Prein. Maeven Prein est donc né au sein d’une famille qui fait du cirque depuis de nombreuses générations.

La première fois que Maeven s'est produit sur une piste de cirque, il n’était que seulement âgé de 9 ans. À l’époque, il commença en tant que jongleur, mais au fil de sa carrière, il présenta en réalité cinq numéros différents. Par la suite, il se spécialisa dans le numéro qui mettra en lumière son cirque : la roue de la mort. Accompagné de son frère, il forme le duo Prein Brothers.

Malgré leur jeune âge et la jeunesse de leur numéro, le public reste captivé par cette performance sur le fil du rasoir. À l’hiver 2023, ils se produisirent sur la mythique pelouse de Reuilly sous le plus grand chapiteau rond du monde, celui du Cirque Mondial de Maxime Kerboua. Placé en numéro final, à chaque représentation, le public hurle, en redemande. Une chose est sûre : l’avenir s’annonce radieux pour les Prein Brothers et pour Maeven Prein.

\section*{\textit{Frédéric Edelstein, l’homme aux douze lions blancs}}
\phantomsection
\addcontentsline{toc}{section}{Frédéric Edelstein, l’homme aux douze lions blancs}

Frédéric Edelstein est né le 30 juillet 1969 à Lyon et il est le fils de Gilbert Edelstein, à l’époque commercial et futur propriétaire du Cirque Pinder. En 1983, alors qu'il n'a que 14 ans, Frédéric voit son père devenir propriétaire du Cirque Pinder. Frédéric se ravit de cette nouvelle vie qui s’offre à lui et très vite se passionna pour le cirque, les animaux et plus particulièrement pour les fauves.

Loin de faire l’unanimité, son père ne voit pas d’un bon œil la nouvelle passion de son fils, toujours à flâner avec les dresseurs près des cages. La carrière de Frédéric Edelstein est le fruit de sa passion développée depuis des années, de son courage, mais également d'un merveilleux coup du sort qui changera sa vie entière.

Un jour, alors que son père est en déplacement, il se dispute avec le dresseur engagé du cirque. Le belluaire décide de ne plus présenter son numéro au spectacle et le Cirque Pinder se retrouve alors sans numéro de fauves. C’est à ce moment que Frédéric sent la chance tourner pour lui. Même s'il n’a tout juste que 14 ans, il décide de faire monter la cage dans l’après-midi, de répéter le numéro que le dresseur présentait habituellement, il le connaissait par cœur avec le temps. C’est ainsi que le soir même, il décida de présenter en cachette de son père un numéro composé de sept tigres.

En l’apprenant, son père furieux lui aurait dit : « \textit{je ne t’ai pas mis au monde pour que tu te fasses bouffer par un tigre !} » Malgré la colère de son père, Frédéric sait désormais qu’il veut faire dresseur et rien d’autre. Il décide alors à quelques mois du baccalauréat de le rater et devient dresseur sous la tutelle de Wolfgang Holzmaïr et Dicky Chipperfield, deux grands dresseurs.

Avec le temps, il deviendra la tête d’affiche du cirque Pinder dont il deviendra également directeur. Il faut dire que la volonté de Gilbert était de faire de son fils le Gunter Gebel-Williams français. Même style de costume, même starification, radio et télévision sont devenues également le quotidien de Frédéric. La différence notable entre Gunter et Frédéric est que le premier est dompteur et le second dresseur, mais nous reviendrons à cette distinction plus tard dans cet ouvrage.

Au cours de sa carrière, Frédéric se rendra célèbre avec un numéro mixte, composé de lions, de lionnes, de tigres et de tigresses, pouvant aller de seize à vingt fauves. Dans la seconde partie de sa carrière, il présenta un numéro composé de 12 lions blancs, qu'il travailla avec son maitre Dicky Chipperfield. Hélas, le Cirque Pinder est placé en liquidation judiciaire en mai 2018 et Frédéric se retrouve sans cirque.

En décembre 2018, il est annoncé au programme du Grand Cirque de Noël Américain de Jean Arnaud, mais sa venue sera annulée pour des raisons techniques, c’est Teddy Seneca qui présentera son groupe de lions. Le 24 décembre de la même année, Frédéric se produit à Nantes au Grand Cirque de Noël Medrano. L’année suivante, il fera quelques dates avec le Cirque Claudio Zavatta de la famille Prein qui présentera deux numéros de fauves à chaque représentation : celui de Didier Prein et celui de Frédéric. Depuis quelque temps, Frédéric a cessé de se produire et vit à Pinderland avec ses fauves, dans l’espoir de repartir un jour sur les routes de France.

Symbole d’une génération, Frédéric fut « l’idole des jeunes passionnés », le symbole de l’enfance pour certains, dont je fais partie. Avec ses interventions dans les médias, sa passion et son professionnalisme, Frédéric a su conquérir le cœur des passionnées de cirque. 

\section*{\texorpdfstring{\textit{Roman \& Laurent, un jeune ventriloque prometteur}}{Roman et Laurent, un jeune ventriloque prometteur}}
\phantomsection
\addcontentsline{toc}{section}{Roman \& Laurent, un jeune ventriloque prometteur}

Roman est né le 17 octobre 2009 à Caen, dans une famille qui ne le prédestinait pas au cirque. Très tôt dans son enfance, il se passionna pour le monde du cirque et voulut en faire sa vie. La carrière de Roman commence lorsqu’il fait une rencontre particulière.

Alors en séjour à Londres, il arpente avec ses parents les rayons d’Hamleys, l’un des plus grands magasins de jouets du monde, lorsqu’il vit une marionnette de perroquet. Elle lui fit de la peine et Roman décida de la prendre. Les parents de Roman ont vite compris que sa voie avait été trouvée en le voyant avec.

Au Noël de l’année suivante, il reçut en cadeau une nouvelle marionnette et cette fois-ci, ce fut un orang-outan. Roman décida de le nommer Laurent, faisant ainsi un jeu de mot simple et efficace : « Laurent outan ». C’est ainsi que le duo Roman \& Laurent fut né. Laurent prendra une place particulière dans la vie de Roman qui est fils unique et le reconnait comme un frère. D’ailleurs, dès le départ, il fut ravi de l’arrivée de ce petit orang-outan, car il trouve que les singes sont très proches des humains et il en joue.

Grand admiratif du travail de Michel Dejeneffe et de sa marionnette Tatayet, lui aussi voulait monter un duo humoristique avec sa marionnette. Cependant, il fut confronté à une difficulté importante : il n’arrivait pas à ventriloquer.

Un jour, il se réveilla avec un mal de gorge très particulier. À vrai dire, il n’avait jamais ressenti cette sensation auparavant. Quelques pas en dehors de son lit et il trébucha en laissant s’échapper un éclat de voix qui ne venait pas de ses cordes vocales, mais de son diaphragme. Pour lui, cette chute fut un déclic, il comprit comment ventriloquer.

Avec détermination, il travailla tous les jours afin de maitriser la ventriloquie et affirme que « \textit{la ventriloquie est comme de la sculpture, c’est une pâte que l’on façonne} ». Il travailla dans l’ombre pendant quatre ans avant de devenir l’un des plus jeunes ventriloques de France.

À l’été 2022, en vacances avec ses grands-parents à Guérande, il réalise son rêve en se produisant pour la première fois sur la piste d’un cirque, celle du Cirque Nicolas Zavatta de la famille Douchet. Il doit le début de sa carrière à Théo Leroy qui décide de le proposer aux propriétaires du cirque qui acceptèrent. C’est de cette manière qu’il se présenta sous le chapiteau du Cirque Nicolas Zavatta à Guérande. Avant de se produire au cirque, il organisait également de nombreux spectacles de rue improvisés dans le but de gagner en expérience pour le jour venu.

De juillet à août 2023, il se produit sur la piste du Cirque Europa de la famille Krosemann. Il se produisit à Cherbourg, Bayeux et dans d’autres villes du nord de la France. Durant cette période, il se présenta également dans d’autres établissements comme le Cirque Rolph Zavatta de la famille Prein ou encore au Cirque Francesco Corbini de la famille Corbini.

Sa plus grande fierté est de voir, lors de son numéro, l’émerveillement du public et le sourire des enfants, non pas pour lui et son talent, mais pour Laurent, sa marionnette, sur laquelle tous les yeux sont rivés.

\section*{\textit{Natalya Jigalova, l’étoile du trapèze ballant }}
\phantomsection
\addcontentsline{toc}{section}{Natalya Jigalova, l’étoile du trapèze ballant}

Natalya Borisnova Vul, plus connue sous le nom de Natalya Jigalova, est née le 21 juillet 1970 à Chișinău, capitale de la Moldavie. La carrière de Natalya Jigalova débute lorsqu’en 1985, elle postula et fut admise à l'école de cirque d'État de Moscou. Dans cette prestigieuse école, elle se forma aux différents arts de la piste et y rencontra son futur mari Andrey Jigalov, futur célèbre clown. 

Dans le cadre de la préparation de son diplôme, avec l’aide de Victor Formine, elle élabore un numéro de trapèze ballant. Un numéro novateur dont seulement quelques personnes avaient le secret et surtout la technique.

En 1989, Natalya a 19 ans, est fraîchement diplômée de son école et est prête à conquérir les plus grandes pistes d’Europe. Enfin, en principe, car en réalité, à la fin de ses études, elle se maria dans la foulée avec Andrey Jigalov et tomba enceinte, ce qui repoussa le début de sa carrière d'artiste. 

Ce qui rendait son numéro de trapèze ballant particulièrement intéressant est qu'elle avait imaginé un système de poulie de manière que la hauteur de son trapèze pouvait être variable. Grâce à cette innovation, elle put commencer son numéro au sol, y inclure de la danse pour ensuite évoluer dans les airs, le tout sans aucune longe de sécurité. Cette différence permit à Natalya Jigalova de se distinguer des autres trapézistes qui devaient monter au trapèze avant le début de leurs numéros.

Le travail de Natalya Jigalova est récompensé une première fois en 1996, lorsqu’elle remporte une médaille d'argent au festival mondial du cirque de demain qui s’est déroulé au Cirque d'Hiver Bouglione, à Paris. Remporter un prix dans un festival est pour un artiste de cirque la garantie de décrocher des contrats dans les établissements les plus prestigieux du monde. L'enjeu est donc de taille. Avec cette récompense, sa carrière décolla.

Par la suite, elle se produit au Cirque Knie, le cirque national suisse ; au Cirque Roncalli, en Allemagne ; et au Österreichische National-Circus Louis Knie, le Cirque National d’Autriche Louis Knie. Mais également dans des théâtres de variétés, comme au Palais Royal de Kirrwiller. La chance commence à sourire à Natalya Jigalova et en 2003, elle présente son numéro sur la piste du plus célèbre festival de cirque du monde, celle du festival international du cirque Monte-Carlo. À cette occasion, on lui décerne le prix du Cirque de Budapest. Si tout sourit à Natalya Jigalova sur le plan professionnel, sur le plan personnel, c'est un peu plus complexe puisqu’entre-temps, elle se sépare de son mari Andrey Jigalov.

Désormais mère célibataire et connaissant la précarité de la vie d’artiste, elle décide de prendre sa retraite des pistes et de reprendre des études. Elle obtient alors un diplôme de psychothérapeute, qui ne lui servira que très peu, puisqu'elle regagnera vite le monde du cirque dans une toute nouvelle fonction.	

Elle décide d’accepter l’offre de Maskim Nikouline et devient régisseuse de piste du Cirque Nikouline, le plus célèbre cirque russe. Elle se plait très vite dans ce poste et ses expériences dans les différents établissements d’Europe en font une régisseuse active, professionnelle, un véritable élément moteur du cirque. De 2016 à 2018, elle est régisseuse de piste du festival du cirque du Val-d'Oise. Hélas, on lui diagnostiqua tardivement un cancer du côlon dont elle ne put guérir. Elle rendit son dernier souffle le 13 juin 2022, à Moscou, à l'âge de 52 ans. Après un hommage reçu au Cirque Nikouline, elle fut enterrée au cimetière Khovansky, à Moscou.

\section*{\textit{Henri Dantès, le dompteur du plus grand cirque du monde }}
\phantomsection
\addcontentsline{toc}{section}{Henri Dantès, le dompteur du plus grand cirque du monde}
Heinrich Honvehlmann, plus connu sous le nom d’Henri Dantès, est né le 17 août 1932 à Datteln en Allemagne dans une famille d’industriels et par conséquent, rien ne le prédisposait à la prestigieuse carrière qu’il aura au cirque. Certains hommes sont arrivés au cirque par passion, d’autres par hasard et d’autres par amour pour une femme.

Alors que le Cirque Bouglione avait planté son chapiteau à Munich, Henri Dantès y rencontra une trapéziste, dont il tomba éperdument amoureux et décida de la suivre. C’est ainsi qu’en 1952, Henri Dantès partit en tournée avec le Cirque Bouglione.

Si à ses débuts, il fut le garçon de cage de Firmin Bouglione, très vite, il décela en lui un potentiel rare en voyant la passion qu’il avait pour les fauves. C’est ainsi qu'il le prit comme élève et lui apprit le noble art de la dompte. Il existe une anecdote plutôt cocasse concernant le début de sa carrière et c’est Henri lui-même qui la raconta dans un documentaire de 1992 réalisé par Eric Sandrin, aujourd’hui malheureusement introuvable. Devant la caméra, il anéantit le mythe du dresseur sans peur et avoue qu’au début de sa carrière, il était tétanisé par la peur à l’idée d’entrer en cage. Il avoue même avoir quelquefois pleuré.

Autre anecdote, l’origine de son nom de piste. Heinrich Honvehlmann a décidé de s’appeler Henri Dantès pour plusieurs raisons. Henri est une francisation de son prénom et Dantès fait référence à Edmond Dantès, le héros de la célèbre œuvre d'Alexandre Dumas : \textit{Le Comte de Monte-Cristo}.

Durant sa carrière, il se spécialisa dans les animaux sauvages. Il présenta donc plusieurs groupes de fauves avec des tigres, des lions, des panthères, mais également des ours. Sa grande spécialité resta cependant les tigres et les lions.

L’un des numéros qui le rendit célèbre était composé d’un groupe de lions mâles. Pour conclure son numéro, il effectuait un exercice particulièrement dangereux dans lequel chaque lion venait tour à tour s’allonger sur lui.

Son travail de qualité lui fit une réputation dans le monde du cirque. À l'époque, il eut l'opportunité de travailler dans les plus illustres cirques français comme au Cirque Pinder, au Cirque Amar, au Cirque Grüss ou encore au Cirque Jean Richard.

La carrière d’Henri Dantès fut également marquée par des tournages dans différents films. En 1956, il joua la doublure de Burt Lancaster dans le film \textit{Trapèze} de Carol Reed. En 1964, il interprèta le rôle d’Emile Schuman, un dresseur de fauves terrorisé à l’idée de rentrer en cage avec des tigres dans \textit{Le Plus Grand Cirque du Monde} d'Henry Hathaway. En 1966, il tourna un dernier film loin des sentiers dorés du cirque en jouant dans \textit{La Bible} de John Huston.

Henri Dantès fit également plusieurs apparitions dans l’émission \textit{La Piste aux étoiles} de Gilles Margaritis. En 1967, il y présenta notamment un groupe de tigres au Cirque d’Hiver Bouglione, dont les images sont encore visionnables aujourd’hui. En 1972, à l’occasion du 39\ieme~ gala des artistes présenté par Jerry Lewis au Cirque d’Hiver Bouglione, Jean-Claude Brialy devint disciple d'Henri Dantès qui le forma au métier de dresseur. Jean-Claude Brialy présenta ainsi, le temps d'une soirée, le groupe de fauves d’Henri Dantès.

À la fin de sa carrière, il tourna dans de petits établissements comme le Cirque Roger Lanzac dans les années 1990. Il travailla également dans des zoos et s’efforça de transmettre son savoir-faire acquis durant toutes ses années à travailler avec des fauves. Henri Dantès s’éteignit le 28 février 1997 à Bordeaux à l’âge de 64 ans.

\section*{\textit{Michel Palmer, Monsieur Loyal des grands cirques}}
\phantomsection
\addcontentsline{toc}{section}{Michel Palmer, Monsieur Loyal des grands cirques}

Michel Palmer est originaire de Dunkerque. Il est issu d’une famille de la petite bourgeoisie provinciale et a toujours été passionné de cirque depuis son enfance. D’ailleurs, il évoque au sujet de son enfance la joie indicible qu’il ressentait lorsqu’un cirque s’installait dans sa ville. Il s’amusait pendant des heures à présenter les numéros de son cirque miniature qu’il fabriquait avec deux règles et une serviette de toilette.

La carrière de Michel Palmer commença lorsqu’il prit une décision plus que risquée. À l’époque, âgé de 18 ans, il décida trois mois avant de passer son baccalauréat, d’arrêter les études et de devenir le caissier du Cirque Albert Rancy. À cette époque-là, il fit la rencontre d’Arlette Grüss, une rencontre qui allait marquer sa vie de manière décisive.

Après cette première expérience au cirque d’un an, il revint vivre chez ses parents. Sur leurs conseils, il décida de reprendre ses études en passant son baccalauréat avant de faire des études supérieures de comptable. Après l’obtention de son diplôme, il devint alors le comptable du Cirque Jules Verne d’Amiens.

En 1985, le Cirque Albert Rancy est désormais fermé. Arlette Grüss décida de monter son cirque avec Georgika Kobann et fit appel à Michel Palmer. Elle lui proposa un poste au service administratif et publicitaire du Cirque Arlette Grüss. C’est une opportunité inédite qui s’offrit à Michel Palmer, qui accepta sans savoir que ce sera grâce à Arlette Grüss qu’il deviendra plus tard Monsieur Loyal.

Un jour, Arlette Grüss rencontra des imprévus et se retrouva sans Monsieur Loyal. À l’époque, il est inconcevable d’offrir un spectacle sans présentateur pour accueillir le public et présenter les numéros. Elle décida de demander à Michel de présenter le spectacle, car elle trouvait qu’il avait une belle voix. Il présenta une première fois le spectacle et Arlette Grüss fut si satisfaite qu’elle décida de le garder à ce poste qui lui allait si bien. De cette manière, il fut Monsieur Loyal du Cirque Arlette Grüss pendant 23 ans. En 2007, il quitta sa fonction de présentateur, mais travailla toujours dans les bureaux du Cirque Arlette Grüss jusqu’en 2010. À cette époque, c’est Claude Brunet qui prit sa succession durant un an, le temps que Kevin Sagau se forma. Il quitta ensuite le cirque pour rentrer à Amiens.

Michel Palmer participa aussi à de nombreuses reprises à des spectacles du Cirque Medrano et du Grand Cirque de Saint-Pétersbourg. Il tint également le poste de conseiller artistique du Cirque Medrano. Il fut Monsieur Loyal du festival mondial du cirque de demain qui se tint au Cirque d’Hiver Bouglione. Pour cette édition du festival, le thème était les Monsieur Loyal et Michel Palmer représentait la France. C’est ainsi qu’il se produisit pour la première fois de sa carrière au Cirque d’Hiver Bouglione.

En 2011, il fut sollicité par la famille Bouglione qui lui proposa de succéder à Sergio, Monsieur Loyal du Cirque d’Hiver Bouglione depuis 1965. Initialement, Les Rois du Cirque lui proposèrent un poste provisoire qui devait durer trois semaines. Mais lorsque Sergio prit sa retraite des pistes en 2012, Michel Palmer prit définitivement sa succession. Encore aujourd’hui, il présente les plus beaux numéros du monde au sein du plus beau cirque stable du monde.

À partir de 2019, il présenta chacune des éditions du festival du cirque de Bayeux, un des plus prestigieux festivals de cirque français. En mars 2023, pour ses quarante ans de carrière et ses 35 ans en tant que Monsieur Loyal, on lui décerna un loyal d’or lors de la 10\ieme~édition du festival du cirque de Bayeux. En 2023, il fut également Monsieur Loyal de la 22\ieme~édition du festival international des artistes de cirque de Saint-Paul-lès-Dax. Michel Palmer fut fait chevalier de l’ordre des arts et des lettres en juillet 2023 et aujourd’hui, il occupe également le poste de conseiller artistique au Cirque Jules Verne d’Amiens. 

\section*{\textit{Roger Falck, la fierté française à Monte-Carlo }}
\phantomsection
\addcontentsline{toc}{section}{Roger Falck, la fierté française à Monte-Carlo}

Roger Falck est né en 1989, à Bordeaux, dans une famille circassienne depuis sept générations originaires d’Allemagne, la famille Falck. Roger Falck fut toujours passionné par les fauves, il fit d’ailleurs sa première entrée en cage, en 1994, à l’âge de 5 ans. En 2003, à l'âge de 14 ans, Roger Falck présenta son premier numéro de fauves. Au cours de sa carrière, il éleva dans sa caravane huit tigres blancs, ce qui est extrêmement rare. 

Roger Falck est reconnu pour être l'un des rares dresseurs à ne pas travailler avec de la viande en guise de récompense, que ce soit pour le dressage ou pour le spectacle. Il a essayé, dans son approche, de favoriser au maximum l’intelligence des fauves en utilisant uniquement la parole. Avec cette méthode, il a réussi à valoriser l’intelligence de ses fauves qui le comprenaient à la parole. Il est également connu pour être l’un des seuls dresseurs de France à avoir fait marcher deux tigres sur les pattes arrière en même temps, un exercice très complexe à exécuter.

Au fil de sa carrière, il gagna de nombreux prix dans des festivals de cirque. Il remporta notamment le prix du musée du Cirque, le prix spécial Jean Richard ainsi que le prix du Bretagne circus. En 2008, il remporta, lors du festival international du cirque de Massy, un chapiteau de cristal.  En 2009, Roger Falck entra dans l’histoire en remportant un clown de bronze lors de la 33\ieme~édition du festival international du cirque de Monte-Carlo. Le prix, aussi prestigieux soit-il, lui fut remis par la princesse Stéphanie en personne.

En 2012, Roger Falck annonça vouloir monter un nouveau numéro appelé « La roue de la mort » composé de tigres et de lions. Finalement, son dernier numéro fut composé de 13 fauves avec des tigres blancs, des lions blancs, des tigres golden tabby et des lionnes.

Il est également important de noter qu’au cours de sa carrière, il fut en piste avec tous les animaux de son cirque, le Cirque La Piste aux Étoiles. Le cirque de sa famille avait une belle ménagerie, il put donc présenter, en plus de son groupe de fauves, des éléphants, des zèbres, des chameaux et des dromadaires. D’ailleurs, Roger Falck eut la chance de participer une seconde fois au festival international du cirque de Massy avec son troupeau d’éléphants.

\section*{\textit{Joseph Bouglione, d’écuyers hors pair à directeur artistique}}
\phantomsection
\addcontentsline{toc}{section}{Joseph Bouglione, d’écuyers hors pair à directeur artistique}
Joseph Jacques Bouglione est né le 26 novembre 1960 à Paris. Il est le fils d'Émilien et de Christiane Bouglione et représente la sixième génération de la plus fameuse famille circassienne française, la famille Bouglione. Dans la suite de cette section, nous l’appellerons simplement Joseph Bouglione. C’est à l’âge de 12 ans que Joseph Bouglione commença à s’initier aux différents arts de la piste. Toujours dans cette tradition qui lie performance et polyvalence, il apprit autant le jonglage, l’acrobatie et l’équitation, mais également la pratique du piano et de la trompette. Il est d’ailleurs un grand amateur de musique jazz.

En parallèle de sa période d’apprentissage, il engagea sa carrière au cirque en tant que garçon de piste. Les premiers numéros qu’il présenta en piste furent un numéro d’acrobatie et un numéro de présentation de poneys. À l’instar de son père, Émilien Bouglione, il se fit connaitre en tant que maitre écuyer talentueux plus tard dans sa carrière. Néanmoins, le numéro qui le rendit célèbre fut un numéro de fil de fer. Il commença à travailler son numéro de fil de fer lorsqu’il avait 16 ans. Il fallut attendre l’année 1978, lors de la tournée du Cirque Bouglione pour que le jeune Joseph, âgé à l’époque de 18 ans, se présente pour la première fois comme fil-de-fériste.

Depuis cette époque, il se fit connaitre pour être un fil-de-fériste brillant et travailla dans de nombreux cirques, cabarets et music-halls. Fait notable dans sa carrière, il travailla au Cirque Roncalli, l’un des plus célèbres cirques d’Allemagne, pendant quinze ans.

En 1984, en récompense pour son travail sur les plus grandes pistes du monde, Jack Lang, à l’époque ministre de la Culture, lui décerna le grand prix national du cirque. L’année suivante, il participa au 11\ieme~festival international du cirque de Monte-Carlo, en 1985. D’ailleurs, il présenta son remarquable numéro de fil de fer lors de la 20\ieme~édition du festival international du cirque de Monte-Carlo, pour sa seconde participation en 1996.

En 1999, la famille Bouglione relança leur activité de prédilection et à cette occasion, Joseph devint le directeur artistique du Cirque d'Hiver Bouglione. Cette année-là, il signa alors sa première création, nommée \textit{Salto}, qui fut un franc succès. Depuis 1999, il signe chaque année les créations originales du Cirque d’Hiver Bouglione. Parmi ses créations, on compte \textit{Piste}, \textit{Trapèze}, \textit{Le cirque} qui fête les 150 ans du Cirque d'Hiver et \textit{Fantaisie} qui fête les 170 ans du Cirque d’Hiver. C’est un sans-faute pour Joseph Bouglione depuis \textit{Salto} et chacune de ses créations est une pièce unique qui vient marquer sa carrière de directeur artistique, mais aussi tous les artistes et le personnel qui y contribuent.

La reprise des spectacles au Cirque d’Hiver permit également à Joseph de se mettre en piste en tant que maitre écuyer hors pair. En 2003, pour la création \textit{Voltige}, il rendit hommage à l’illustre carrière de son père en présentant son fameux numéro de la poste. De plus, Il présenta des numéros de cavalerie en liberté : en 2005, pour \textit{Audace} ; en 2008, pour \textit{Étoiles} et en 2009, pour \textit{Festif}. À titre personnel, j’ai eu la chance de le voir présenter un groupe de poney lorsqu’il dut remplacer momentanément Regina Bouglione lors de \textit{Fantaisie}. L’aisance en piste, les gestes de chambrière exécutés avec adresse, sa prestance et son sourire m’ont particulièrement touché. Joseph Bouglione est un grand maitre écuyer.

Il fallut peu de temps pour que son talent de directeur artistique soit reconnu internationalement. Au cours de sa carrière, il produisit de nombreux spectacles en Inde, en Hollande, au Royaume-Uni et en Espagne. De 2005 à 2007, il fut directeur artistique du Cirque Tihany et depuis 2023, il produit les spectacles du Weltweihnachts Circus Stuttgart. Il produit également les spectacles de cirque pour le parc d'attraction Europa-Park, en Allemagne. Mais ce n’est pas le seul lien qu’il possède avec l’Allemagne puisqu’il collabore pareillement avec le Cirque Roncalli de manière régulière depuis 1985. Aux États-Unis, il est directeur artistique du Cirque Vasquez depuis 2013.

Même si Joseph Bouglione est un homme passionné de cirque, il ne s’interdit pas de produire des créations de cabarets, de music-halls et de variétés. C’est ainsi qu’il produisit des spectacles en Allemagne pour le Roncalli’s Apollo Variété de 2011 à 2018. Le monde du cabaret français veut également s’offrir l’expertise de Joseph, tel que le Lido, un grand cabaret parisien, avec lequel il collabore. Fait notable dans sa carrière, en 2015, il fut le premier homme à produire un spectacle de cirque traditionnel sous chapiteau à Sanya en Chine. En 2021, il assura la direction artistique de \textit{La H Arena fait son cirque} qui fut un succès. Une deuxième édition eut lieu en 2022 avec comme nouveau nom \textit{La H Arena refait son cirque} avant de revenir sous son nom originel pour une nouvelle édition en 2023. Il sera par ailleurs chargé de la direction artistique du spectacle \textit{Le Grand Cirque de Vendée} du 30 novembre au 1 décembre 2024.

En 2011, avec Soffia Morghad ils donnèrent naissance à Juliano Bouglione qui représente la septième génération de la dynastie Bouglione. Aujourd’hui, Juliano commence sa carrière d’artiste en tant que fratoche et effectue des apparitions remarquées à la batterie. Homme et artiste accompli, Joseph a vu son travail récompensé à de multiples reprises. En 2003, il fut fait chevalier de l’ordre national du Mérite et en 2019, en reconnaissance pour son illustre carrière, la fédération mondiale du cirque lui décerna le prix de l’ambassadeur du cirque. Ce prix, aussi particulier soit-il, prouve qu'il est l’une des personnalités qui s’évertue à faire rayonner le cirque, son histoire et son patrimoine. Cette volonté de promouvoir cet art se concrétisa avec un projet qu’il dévoila il y a peu de temps aux yeux du grand public.

En 2023, Joseph Bouglione déclara avoir le projet d'ouvrir une école de cirque à Lizy-sur-Ourcq. Ce projet naquit lors du confinement, lorsqu’il entreprit de faire du rangement dans le terrain Bouglione d’une surface d’un peu plus de mille mètres carrés. Cette ville, il ne l’a pas choisie par hasard puisque c’est dans cette ville que sont enterrés, depuis 1897, les ancêtres de la famille Bouglione. Lizy-sur-Ourcq, c’est également la ville de la base arrière de la famille Bouglione. 

Cette école propose des stages d’initiation au cirque pour enfant et pour adulte, mais ce n’est pas le seul service qu’elle offre. En effet, selon les déclarations de Joseph, cette école sera aussi le lieu de répétition d’artistes de cirque professionnels qui chercheront un endroit où se dépasser et créer de nouvelles choses. La particularité de cette école est qu’elle conserve les traditions du cirque en possédant une vraie piste sur laquelle enfant et adulte peuvent s’adonner aux différents arts du cirque. Les disciplines enseignées incluent la jonglerie, le trapèze fixe, le tissu aérien, la spirale aérienne, la gymnastique, le fil de fer, la contorsion, le rola bola, le cerceau, l’acrobatie et les assiettes chinoises, etc. Un projet ambitieux qui rencontre déjà un grand succès dans sa région en initiant de nombreuses écoles aux arts du cirque. Les inscriptions devraient commencer en septembre 2024. Joseph a d’ores et déjà annoncé que si la demande se fait grande, l’école pourra également accueillir de nouvelles disciplines comme du yoga, de la méthode Pilates ou encore de la danse.

Joseph Bouglione est un homme qui aura donné et qui donne encore toute son énergie pour ce qui le fait vibrer : le cirque. Ces efforts sont récompensés tant par le public qui applaudit et attend chacune de ses créations, mais aussi par la profession qui voit en lui un réel ambassadeur du cirque. C'est un homme souriant, mais qui possède le goût du travail bien fait et veille sur la piste du Cirque d’Hiver comme sur ses enfants. Véritable perfectionniste, Joseph Bouglione n’a d’ambition qu’à la hauteur de son panache.

\section*{\textit{Alfred Beautour, l’homme aux léopards}}
\phantomsection
\addcontentsline{toc}{section}{Alfred Beautour, l’homme aux léopards}

Alfred Beautour est né le 23 septembre 1924 à Bourg-Achard en Normandie. Il est l’héritier d’une des plus anciennes familles circassiennes de France : la famille Beautour. Son père, Émilien Stanislas Adolphe Beautour, fut un banquiste qui tourna avec son cirque sous différentes enseignes dans l’entre-deux-guerres, parmi lesquels on compte le Cirque Australien, le Cirque Canadien et le Cirque des Alliés. Le Cirque des Alliés devint d'ailleurs après la Seconde Guerre mondiale le Britannique Circus. En 1959, il tourna sous le nom Cirque Continental.

Émilien Stanislas Adolphe Beautour fut plus connu sous le nom d’Henri Beautour et eut trois fils : Lucien, Émilien et Alfred. Ses trois fils se firent connaitre comme dresseurs. Lucien présenta des chimpanzés avec sa femme en formant le duo Luc et Bella, Émilien fut plus connu sous le nom de Tarass Boulba et fut dresseur de fauves comme son frère Alfred, aussi appelé Fredo Beautour.  Si la spécialité de Tarass Boulba fut les tigres, celle d’Alfred Beautour fut les léopards, une espèce rarement présentée au cirque et qui est en réalité très difficile à éduquer.

Alfred se maria avec Yolande Prin, le 10 novembre 1951, elle aussi héritière d’une grande famille de cirque : la famille Prin. C’est à cette époque qu’Alfred, sous le nom de piste Fred Jackson, commença sa carrière de dresseur en présentant un groupe de lions au Britannique Circus. À l’époque, vêtu d’un pagne de Tarzan, il présenta ses fauves en férocité, une méthode de dressage surannée et abjurée à ce jour. L'année 1955 marqua un tournant décisif dans sa carrière, car c’est cette année qu’il acquit son premier groupe de léopards. Rapidement, Alfred se fit connaitre, grâce à ses nouveaux pensionnaires tachetés, comme un dresseur talentueux. Ainsi, Alfred Beautour connaitra une grande carrière de dresseur de fauve entre 1956 et 1989. Il fut très connu en France, mais également dans toute l’Europe.

Dès lors, il quitta son cirque familial pour se produire dans différents pays d’Europe, notamment en Espagne et en Allemagne. Même si sa spécialité resta les léopards, il présenta occasionnellement d’autres groupes de fauves. En 1960, il présenta d’ailleurs un groupe de tigres dans un cirque en Espagne. En 1987, c’est la consécration pour Alfred auquel on décerna le prix Henry Thétard et le prix du Club du Cirque lors de la 12\ieme~édition du festival international du cirque de Monte-Carlo. Fait également notable dans sa carrière, il participa à l’émission \textit{La Piste aux étoiles} avec son groupe de léopards en 1972.

En 1989, alors âgé de 65 ans, Alfred décida de prendre sa retraite des pistes. Il se sépara de son groupe de léopards, qu'il vendit au Circo Weglions en Italie. C'est Pascale Forminaso qui les présenta.

Le 17 février 2014, il décéda à l’âge honorable de 90 ans après avoir eu une brillante carrière et avoir gagné le respect de la profession.

\section*{\textit{Dani Lary, et le piano volant}}
\phantomsection
\addcontentsline{toc}{section}{Dani Lary, et le piano volant}

Hervé Bittoun, plus connu sous le nom de Dani Lary, est né le 9 septembre 1958 à Oran en Algérie. Il n’est pas directement lié au monde du cirque, mais son parcours méritait d’être dans cet ouvrage et que serait le cirque sans un peu de magie ? La vie de Dani Lary débuta sous le soleil brulant d’Algérie, toutefois il ne le connaitra que très peu de temps. En effet, alors en pleine guerre d’Algérie, la menuiserie de son père, qui était la plus grande d’Oran, fut détruite. Le père de Dani, ne sachant subvenir aux besoins de sa famille après cette tragédie, décida d'écrire au général de Gaulle. On ne connaît pas le contenu exact de cette lettre, néanmoins selon son fils, elle aurait commencé par : « Je vous ai compris, maintenant, c'est vous qui allez me comprendre ». Dans cette lettre, on peut imaginer que le père de Dani fit part de son désarroi au vu de sa situation. Quelque temps plus tard, la famille Bittoun reçut une réponse qui leur disait : « Venez à Colombey-les-Deux-Églises, on n'a pas de menuisier ». Ainsi, après la destruction de la menuiserie et la réponse du général de Gaulle, ils furent contraints de quitter l’Algérie en 1962. Désormais arrivés en France, ils s’installèrent à Rennepont. Ce fut dans cette première ville que le père de Dani devint menuisier. Par la suite, ils s’installèrent dans un HLM de Bourg-de-Péage dans lequel ils vécurent à huit avec leur grand-mère.

En réalité, la passion de Dani Lary pour la magie commença alors qu’il était encore enfant. Un jour, lorsqu'il avait 8 ans, il vit le fameux numéro du journal reconstitué interprété par le magicien Henri Kassagi. Dans ce numéro, le prestidigitateur déchirait successivement un journal qu’il réussissait, in fine, à reconstituer. Ce numéro, considéré comme un classique de la magie, subjugua l’enfant qu’il était et lui donna la passion de la magie. Les parents de Dani ont très vite compris que la magie serait la nouvelle passion de leur fils et décidèrent de lui offrir au Noël de ses 8 ans la boite de magie d’Henri Kassagi. Dani fut empli d’une joie que seuls les passionnés peuvent comprendre. D’ailleurs, dans son numéro \textit{Rêve de père Noël}, il explique ce jour si particulier en ces mots : «\textit{ Mesdames et messieurs, j’aimerais vous parler d’un personnage extraordinaire qui a bouleversé ma vie : le père Noël. À l’âge de 8 ans, je rêvais de devenir magicien. J'ai alors commandé une boite de magie et le 25 décembre, voilà ce que j’ai trouvé dans mes souliers. Je m’en rappelle comme si c’était hier, il neigeait plus que les autres Noëls ce jour-là, et j’étais le petit garçon le plus heureux de la terre. Claire rêvait de devenir danseuse, aujourd'hui, Claire et moi avons réalisé notre rêve, et c’est un petit peu normal que nous dédions au père Noël ce numéro, regardez. }» Tandis que la plupart des enfants qui reçoivent une boite de magie s’amusent quelques heures et se lassent vite des arcanes de la magie, Dani quant à lui, apprit tous les tours de la boite par cœur. 

La petite carrière de Dani Lary commença lorsqu’il réussit à convaincre sa maitresse d’école de faire un numéro de magie pour la fête de l’école. À l’époque, c’est cette même institutrice qui l’interdit de chanter à la chorale à cause de sa voix de crécelle, ce qui peina le jeune enfant. C’est de cette manière qu’il réussit à se produire pour la première fois à la fête de son école à Bourg-de-Péage. Cette première expérience fut un succès, il réussit à séduire petits et grands grâce à son talent et sut désormais ce qu'il voulait faire de sa vie : magicien.  

Plus tard, c'est grâce à sa sœur Brigitte Bittoun qu’il décida d’avoir un nom d’artiste pour devenir un vrai artiste. Le petit Hervé aimait beaucoup « Dani Lan » qui était le nom de la boutique de sa tante Gaby, couturière qui travaillait avec Alain Manoukian. Sous les conseils de sa sœur, il ne choisit pas « Dani Lan » mais « Dani Lary » qui lui semblait un nom d’artiste court et efficace. 

En 1976, pour ses 18 ans, son père voulant marquer le coup lui demanda ce qui lui ferait plaisir. Le jeune Dani déjà plein d’ambition, lui répondit qu’il voulait son ancien camion, un tube Citroën. Son père ne comprenait pas pourquoi son fils voulait tant ce camion qui ne fonctionnait presque plus, mais Dani avait une belle idée derrière la tête. Dani apprit alors la mécanique et remit en état le camion que son père lui offrit. Par la même occasion, il décida également de le peindre en violet et d’y inscrire : « International Magic Show Dani Lary ». C’est avec son camion rempli de son matériel de magie qu’il décida de partir en Espagne, faire la tournée des boites et des campings. Au départ, il se présenta aux responsables en tant que jeune magicien passionné qui offrait ces services gratuitement. Les campings acceptèrent et c’est ainsi que Dani Lary présenta des spectacles gratuitement les après-midis et les soirées dans des campings espagnols. En réel passionné, il posa un chapeau et reçut les pourboires de qui voulut bien lui donner. Mais il insistait bien sur le fait qu’ils n’étaient pas obligés de donner et que le simple fait de se produire devant eux lui faisait plaisir. 

Au début, il se produisit de manière bénévole puis les campings commencèrent à le rappeler en le payant cette fois-ci. Il se retrouva donc à recevoir des cachets pour ses spectacles en plus des dons que le public continuait à lui faire. Fier de son succès ibérique, il décida de rentrer en France afin de retrouver sa famille. Sur son chemin, il arriva à Toulouse et tomba sur le congrès de la Fédération Française des Artistes Prestidigitateurs, auquel il rêvait d’assister. Néanmoins, il n’avait pas assez d’argent pour se payer l’entrée du congrès et voulait absolument préserver sa recette de l’été. Il décida alors de s’inscrire au concours de magie organisé pour cette occasion qui donnait le droit à l’accès au congrès gratuitement. Après quelques négociations, il réussit à s’inscrire et présenta lors du concours son numéro de pierrot. Son numéro se fit remarquer par sa poésie, sa technicité et son caractère visuel. Le public du congrès fut conquis et lui offrit sa première standing ovation. À l’annonce des délibérations du jury, Dani Lary entra dans l’histoire en remportant le grand prix de magie. Il devint champion de France de magie à 18 ans. Anecdote importante : le jury de l’époque, ayant jugé Dani avant-gardiste, décida de ne décerner ni deuxième ni troisième prix. Au cours de sa carrière, il remporta également une baguette d’argent lors de son invitation au Monte-Carlo Magic Stars. 

Rentré de son escapade espagnole, il se mit à travailler à La Charrette, un restaurant café-théâtre à Romans-sur-Isère. Ce fut dans ce lieu qu’il fit la rencontre de Dave qui devait chanter quelques jours après à La Charrette. Après son concert, Dave devait partir à Toulon pour rejoindre une croisière en tant qu’artiste invité. Le producteur de Dave en discuta avec le directeur de La Charrette et profita pour lui faire part d’un imprévu. Le magicien qui devait se trouver sur la même croisière que Dave s’était cassé la jambe. Christian Vincent, le directeur de La Charrette, lui répondit que s’il cherchait un magicien, Dani serait parfait pour lui. C’est ainsi que Dani partit en croisière avec Dave pour remplacer son confrère magicien. Au départ, son remplacement devait durer seulement une semaine, mais face à son triomphe, la directrice de la croisière Monique Rose décida de le garder 1 mois. Lorsque le magicien titulaire fut rétabli, Monique Rose décida de garder Dani Lary et ses numéros inédits, au total, il resta sur cette croisière 6 ans. 

Au bout de 6 ans, la routine et l’envie de découvrir autre chose se fit sentir et Dani, en compagnie de son assistante, décida de s’installer à Berlin afin de travailler pour un cabaret français qui se nommait « La vie en rose ». Il s’y produisit 4 ans avant de rencontrer Pierre Meyer. Sous la recommandation de Jack Doll, Pierre Meyer engagea Dani dans son établissement. Ce fut une opportunité en or pour Dani qui présenta de nombreux numéros sur la grande scène qu'on lui accordait. Il présenta son numéro de pierrot, son numéro de la boule, mais pas seulement. C’est également à cette époque qu’il inventa son numéro qui représentait un tableau des années cinquante dans lequel il faisait disparaître une Chevrolet Corvette. Pierre Meyer fut si satisfait de son nouveau magicien qu'il lui fit signer un contrat de 2 ans. De plus, c'est à cette époque qu’il fit une rencontre importante pour sa carrière.

En 1990, Christian Fechner, un grand producteur, vint voir le spectacle de Dani accompagné de Siegfried and Roy. À l’époque, ce duo de magiciens régnait sur le monde de la magie et se produisait au Mirage de Las Vegas. Ils furent étonnés du talent de concepteur de tour de Dani qui avait 32 ans et décidèrent de l’engager en tant que responsable des effets spéciaux et des trucages. Ainsi, Dani Lary se fit connaitre dans le monde en tant que talentueux concepteur de tour de magie. Reconnu par ses pairs et par la profession, c’est grâce à son talent de concepteur qu'il entra dans le monde de la télévision. 

En 1998, il fut contacté par Monique Nakachian, directrice de Tavel International Agency qui proposait de nouveaux talents à Patrick Sébastien pour son émission \textit{Le Plus Grand Cabaret du monde}. Elle le contacta pour lui louer un célèbre tour de magie, le panier indien, pensant que le magicien possédait ce tour en stock. Dani trouvait que ce tour était un peu dépassé et lui soumit une idée différente qu’elle accepta. Le jour de l’enregistrement de l’émission, Dani Lary présenta un numéro dont lui seul avait le secret. Sur une table, il y avait un grand carton dans lequel il ligota Jean-Paul Belmondo avant de lui fournir un cutter. Ensuite, il lui annonça qu'il avait quinze secondes pour s’évader de ses propres moyens avant de se faire écraser par un moteur de voiture. Ce moteur qui pesait quatre cents kilos était suspendu au-dessus de lui par une corde. Dani et le public fit le compte à rebours, mais dès la deuxième seconde, le moteur se détacha et pulvérisa le carton. Instantanément, un pompier se jeta sur le carton, l’éventra, le carton est vide, le pompier retira son casque, c'était Belmondo.

Patrick Sébastien fut aussi furieux qu’admiratif du talent de Dani Lary et l’engagea dans son émission par la suite. Durant 20 ans, Dani aura la lourde tâche de finir chaque émission avec un numéro original tous les mois. Parmi ses numéros les plus célèbres, on compte \textit{La veste}, \textit{Le piano volant}, \textit{La Boule}, \textit{Le Fantôme de l’Opéra} et \textit{Le Gloup}.

En outre de sa carrière à la télévision, Dani Lary participa à d’autres projets personnels. En 2004, il devint le parrain des Larys d’or, un concours de magie lui rendant hommage. Au cours de sa carrière, il travailla également avec de nombreux artistes comme Johnny Hallyday, Kamel Ouali, mais aussi pour le cinéma, notamment pour Claude Chabrol en 2007 et Gaël Morel en 2011. La même année, c'est la consécration pour Dani Lary. Son rêve devint réalité lorsqu’il fut à l’affiche de l’Olympia dans lequel il se produisit la première fois le 26 février 2011.

Au cours de sa carrière, il présenta également plusieurs spectacles. Dans ses premiers spectacles \textit{Illusion, tout n’est qu’illusion} et \textit{Le magicien de l’impossible}, il présenta une succession de numéros sans réel lien entre eux. Par la suite, il créa et lança sa première « comédie magicale », une sorte de pantomime de grande qualité. Ainsi fut né \textit{Le Château des Secrets} qui sera une grande réussite dans toute la France. À la suite de ce succès, il créa d’autres spectacles du même style avec \textit{La Clé des Mystères}, \textit{Retro temporis} et \textit{Tic-Tac}. Il interprète actuellement avec son fils Albert Lary, magicien lui aussi, leur nouvelle première pièce de théâtre magique : \textit{Magic Versaire} aussi connue sous le nom de \textit{Comment faire disparaitre son père ?} Il présente en parallèle de cette tournée \textit{Les Folies Barbières} au sein de son atelier magique, un condensé de ses tours les plus célèbres. 

\section*{\textit{Flórián Richter, la culture équestre du cirque magyar}}
\phantomsection
\addcontentsline{toc}{section}{Flórián Richter, la culture équestre du cirque magyar}

Flórián Richter est né le 18 novembre 1977 à Stuttgart en Allemagne. Il est le fils de József et Karola Richter et représente la septième génération de la plus grande famille circassienne hongroise : la famille Richter. Sa famille s’inscrit dans la pure tradition du cirque équestre qui revient à l’essence originelle du cirque traditionnel. 

La carrière de Flórián Richter débuta en 1986. À l’époque, âgé de 9 ans, il dut remplacer sa mère, qui s’était blessée au Japon. C’est ainsi qu’il commença sa carrière de voltigeur qui continua une grande partie de sa vie. En 1998, il se maria avec Edith Folco, elle aussi une héritière d’une famille circassienne italienne. Ensemble, ils auront deux enfants qu’ils élèveront dans la culture du cirque. Ils donnèrent naissance à Kevin Richter en 2001 et à Angelina Richter en 2005, qui représentent tous deux la huitième génération de la dynastie Richter.

La carrière de Flórián Richter fut récompensée de nombreuses fois pour son savoir-faire équestre. En 2003, il reçut le prix Hortobágyi Károly, un prix d’État créé en 1992 qui récompense chaque année le meilleur artiste de cirque hongrois. En 2004, il participa avec sa femme à la 28\ieme~édition du festival international du cirque de Monte-Carlo avec son tableau équestre. Sur le canon de Pachelbel avec sa troupe de danseurs rococos, il présenta ce qui aurait pu être montré à la cour de Louis XIV si le cirque avait existé à cette époque. Pour ce tableau équestre intemporel, le couple Richter remporta un clown d’argent. En 2005, toujours accompagné de sa femme, il décida de monter sa troupe de jockey. Sa troupe fut composée de neuf acrobates hors pair, de sa femme, mais aussi d’un quatuor à corde qui connait la musique magyare mieux que personne. Ensemble, ils travaillèrent pendant 3 ans sur leur numéro avant de connaitre le fruit de leurs efforts. En 2008, à l’occasion de la 32\ieme~édition du festival international du cirque de Monte-Carlo, il présenta le numéro qui le rendra célèbre dans le monde entier : La troupe de jockeys de Flórián Richter. Sur des musiques de Vittorio Monti et de Johannes Brahms, les exercices s’enchainèrent et la troupe fit lever le public à chaque fin de numéro. Pour cette performance, Flórián Richter remporta de nombreux prix, parmi lesquels le tant convoité clown d’or. Ainsi, Flórián Richter devint le premier hongrois, mais également le premier acrobate équestre à remporter un clown d’or de l’histoire du festival. Il remporta également le prix du Blackpool Tower Circus, le prix du Spencer Hodge et le prix Fuentes Gasca.

En 2009, il produisit son premier spectacle équestre sous le nom du \textit{Horse Evolution Show}. Dans ce spectacle, il mit tout son savoir-faire en œuvre afin de valoriser la relation entre l’homme et le cheval. Avec sa première création, il tourna dans toute l’Europe et rencontra un grand succès. En 2012, Flórián ainsi que sa troupe rentrèrent en Hongrie et présentèrent une tournée nationale d’un an. C’est à son retour qu’il reçut la croix d’or hongoise du mérite (Magyar Arany Érdemkereszt).

En parallèle de sa carrière d’artiste, Flórián Richter est également connu pour son engagement pour la préservation du patrimoine du cirque traditionnel. C’est dans cet état d’esprit qu’il décida avec son \textit{Horse Evolution Show} de créer en collaboration avec le Fővárosi Nagycirkusz, l'Eötvös Cirkusz et le Magyar Nemzeti Cirkusz : \textit{La Nuit des Cirques} (Cirkuszok Éjszakáját). Cet événement annuel, qui se produit en juillet à Zamárdi en Hongrie, a pour but de promouvoir les différents arts du cirque. La première édition a eu lieu le 13 juillet 2013 et proposait un spectacle jusqu’à minuit, mais aussi de découvrir les coulisses du cirque, des démonstrations de dressage et des initiations aux arts du cirque. 

En 2014, Flórián Richter se produisit avec sa troupe au Fővárosi Nagycirkusz, le grand cirque de Budapest, pour le spectacle \textit{Circus Classicus}. Sa troupe rencontra un grand succès et y signa une seconde saison en 2015 avec leur spectacle \textit{Les étoiles du cirque hongrois 2} (\textit{Magyar Cirkuszcsillagok 2}). D’ailleurs, au printemps de la même année, Flórián devint le directeur du Fővárosi Nagycirkusz. C’est à cette même époque que le conseil hongrois des arts du cirque le nomma ambassadeur itinérant des arts du cirque hongrois. Sa mission en tant qu’ambassadeur est de promouvoir le patrimoine du cirque et son savoir-faire lors de représentations nationales et internationales. Sa seconde mission est tout aussi importante puisqu’il doit soutenir la formation des artistes de cirque hongrois pour que cette tradition se perpétue.

En 2016, Flórián Richter décida de monter son propre cirque de son côté. En effet, depuis 2012, c’est József Richter Junior, son frère, qui dirige le Magyar Nemzeti Cirkusz. Ainsi, en 2016, naquit le Richter Flórián Cirkusz qui représente aujourd’hui un bel établissement hongrois. Le 12 mars 2016, le Richter Flórián Cirkusz présenta sa première création \textit{Transfomers sur la piste }(\textit{Transformers a porondon}) avant de présenter d’autres créations à succès telles qu’\textit{Animal EXTRÊME} (\textit{Állati EXTRÉM}) en 2017. En 2018, le Richter Flórián Cirkusz présenta sa nouvelle création \textit{Full adrenalin} et \textit{Showtime} pour la saison 2019. 

Fort de son expérience et connu pour ses connaissances équestres, Flórián Richter se produisit dans de nombreux festivals de cirque. Ce fut ainsi qu'en 2019, Flórián se produisit avec son fils au festival du cirque de Budapest avec un numéro de la poste à 20 chevaux, qui remporta le prix du Cirque de la capitale Astana. La même année, lors de la 18\ieme~édition du festival international du cirque Auvergne-Rhône-Alpes Isère, la famille Richter rencontra un grand succès. Elle présenta ses numéros les plus célèbres : la troupe de jockeys, la cavalerie en liberté de Flórián et la poste de Kevin Richter sous la direction de son père.

Ses nombreuses prestations dans les plus grands festivals rendirent Flórián célèbre et lui permirent de se produire dans les établissements les plus réputés. Il se présenta aussi au Cirque de Noël de Stuttgart (Weltweihnachtscircus Stuttgart) de 2021 à 2022 dans lequel il présenta sa cavalerie en liberté. Fin 2022, il participa à \textit{La Grande Fête Lilloise du Cirque} avec sa cavalerie en liberté, mais également avec le numéro de la poste de son fils qu’il dirige toujours. De 2023 à 2024, il se présenta une nouvelle fois au Cirque de Noël de Stuttgart avec un numéro de la poste, mais dont la vedette est cette fois-ci Angelina Richter, sa fille. Après avoir présenté de nombreux numéros avec son fils et avoir rencontré un grand succès, désormais, il présente le même numéro avec sa fille. La boucle est bouclée, la transmission est faite, la tradition est belle.

En 2024, le Richter Flórián Cirkusz présente sa nouvelle création Expérience 2024. 

\section*{\textit{Alfred Court, l’homme derrière la belle et la bête}}
\phantomsection
\addcontentsline{toc}{section}{Alfred Court, l’homme derrière la belle et la bête}

Alfred Court est né le 1 janvier 1883 à Marseille, dans une famille d’industrielle. Son père était dans l’industrie du savon et sa mère était la fille du Marquis de Clapier. Alfred fut donc le cadet d’une fratrie de dix enfants dans une famille relativement aisé. Ainsi rien ne le prédisposait à faire carrière au cirque.

Dans les années 1890, Alfred ainsi que son frère Jules fut envoyés dans une école jésuite du Prado. Dans cette école il fut mauvais élève et relativement turbulent mais se passionna pour la gymnastique avec son frère Jules. D’ailleurs les seuls prix qu’il remporta au cours de sa scolarité fut en dessin et en gymnastique. Avec le temps, Alfred se spécialisa dans les barres fixes. Cette discipline, rarement présenté au cirque, demande une force et de l’habilité que seuls les grands gymnastes possèdent. Avec le temps Alfred devint un barriste talentueux et décida de monter un numéro avec son collègue Alfred Lexton.

Ainsi, le 4 janvier 1899, Alfred Court lança sa carrière au d’artiste de cirque, au Palais de la Jetée à Nice. Alors âgé de 16 ans et sous le nom de piste Alfred Egelton, il monte avec son partenaire barriste le duo Lexton \& Egelton. Ensemble ils rencontrèrent un grand succès et firent leur début au cirque la même année, au Cirque Cristiani. Au terme de sa tournée espagnole, le cirque décida de rentrer en Italie en passant par la France. Le duo Lexton \& Egelton firent leur début au cirque dans l’établissement de la famille Cristiani de passage à Bayonne.

Hélas, la carrière de ce brillant duo dut prendre fin assez vite à la suite d’une mauvaise réception d’Egelton qui l’écarta des pistes pendant un bon moment. Les deux barristes décidèrent de séparer et Lexton décida d’exporter son numéro en Allemagne. Mais la carrière d’acrobate d’Alfred Court ne s’arrêta pas après cette blessure. Après sa convalescence, il travailla quelques mois dans avec une compagnie de théâtre itinérante avant de rentrer à Marseille. 

De retour dans sa ville natale il décida de monter un numéro avec son frère Jules et l’acrobate Féfé Gavazza. Ainsi naquît le trio Egelton’s avec lequel il se firent connaitre. Ensemble, il se produisirent en France au Cirque Pinder de 1905 à 1908 mais également dans tout l’Europe où ils rencontrèrent un grand succès. C’est une époque importante pour Alfred qui rencontra Renée Vasserot, une jeune écuyère du Cirque Pinder qui devint plus tard sa femme.

C’est également à cette également à cette époque que les frères Court décide de monter leur premier spectacle de cirque. Ils créèrent alors le Cirque Egelton et présentèrent leur spectacle dans des constructions de bois éphémère à travers les foires de France. Dans leur spectacle ils présenteront notamment un numéro de vélo acrobatique nommé le looping the loop. A l’époque ce numéro à sensation plut au public mais les frères Court durent l’abandonner à cause des difficultés de transport qu’il engendra. A la fin de la saison 1908, Jules décida de quitter son frère et monta son agence de talents. Entre temps les frères Court rêvèrent toujours de monter le cirque et décidèrent de lever des fonds en attendant le moment propice. 

L’année suivante en 1909, la municipalité de la ville de Marseille accorda aux frères Court l’autorisation de construire un cirque stable de bois sur la place Saint-Michel. Quelques temps plus tard fut alors né le Cirque Egelton, un établissement spacieux qui put accueillir plus de 3000 spectateurs par représentation. C’est au sein de leur temple cirque que les frères Court présentèrent différents spectacles de cirque qui rencontrèrent chacun un succès sans équivoque.

A l’époque, les frères Court ont le souci de toujours présenter de choses nouvelles à leur public. C'est dans cette perspective qu'ils changèrent fréquemment leur spectacle, ce qui ravit leur public mais augmenta considérablement leurs coûts, les poussant ainsi à la faillite en 1912. Ils lancèrent une dernière saison dans leur cirque de bois en 1913, cette fois ci sous l’enseigne Cirque Standard. A l’hiver 1912, les frères Court se produisirent au cirque Cirque Impérial Russe et s’associe avec Ugo Ancillotti.

Avec le temps, le rêve d’avoir son propre cirque itinérant demeura et Alfred décida d’ouvrir un nouvel établissement qui devint plus tard l’un des plus illustre de l’histoire du cirque : le Zoo Zirkus. Entre temps il monte un nouveau numéro de mains à mains avec sa femme et leur disciple Louis Vernet qu’ils nommèrent les Orpingtons. Avec ce nouveau numéro, le trio Orpingtons se produisit au Ringling Bros. and Barnum \& Bailey Circus. à Chicago en pour la première fois en mai 1914. Grâce à leur numéro il réussit à obtenir un contrat de 2 ans. En 1915, il présenta même sur la piste centrale du cirque un numéro de perche. Le voyage d’Alfred ne s’arrête pas au Etats-Unis puisqu’il se produit également avec son trio à Cuba au Circo Pubillones qui leur propose un contrat. C’est ainsi qu’en 1918, Alfred et son trio présente leur numéro de mains à mains et perche. 
C’est après une tournée aux Etats-Unis qu’Alfred Court décida à la fin de l’année 1918 de monter le Circo Europeo et s’associa avec la famille Mijares. Avec le Circo Europeo, Alfred tourna sous un chapiteau de deux mille places et parcourut le Mexique, le Guatemala et le Honduras. C’est grâce à un événement déterminant que le jeune Alfred entra en cage pour la première fois au Circo Europeo. 

Alors en tournée avec son cirque il fut confronté aux aléas qui font la beauté du cirque. Un jour, alors qu’il dut licencier le dresseur de fauve de son cirque qui était alcoolique, le Circo Europeo se retrouva sans dresseur de fauves. C’est à cette époque qu’Alfred qui avait une curiosité nourrit pour les fauves décida de présenter lui-même le numéro de fauve du Circo Europeo. L’aventure d’Alfred en Amérique se conclut à la fin de la Première Guerre mondiale lorsqu’il décida rentrer en France avec sa femme après avoir revendu leur cirque à Don Juan Treviño
De retour en France, Alfred s’associe de nouveau avec Ugo Ancillotti et ouvre un cirque itinérant à Versailles en 1920. Par la suite, ils tournèrent avec cet établissement dans les provinces française. Hélas, cette nouvelle aventure ne dura quelque temps. Ancillotti étant vieux et malade, avant sa mort en 1925, il décida de vendre l’ensemble de ses parts à Alfred. Entre temps Le couple Court décida de reprendre son numéro d’acrobatie avec leur nouveau partenaire Lucien Goddart.

Cependant Alfred n’est pas du genre à abandonner ses rêves et avec le cirque itinérant dont il est devenu récemment le propriétaire il va remettre au bout du jour son projet de cirque. Ainsi il décida de changer de nom d’enseigne en passant de Zoo Zirkus à Zoo Circus, et créa pour la première fois, à Limoge au printemps 1921, le Zoo Circus. Pour l’aider dans son ambitieux, il est rejoint par son frère Jules qui s’occupa de l’administratif du Zoo Circus. Plus que la simple ouverture d’un nouveau cirque parmi tant d’autre, avec son Zoo Circus Alfred importa en France la mode allemande de cirque possédant des ménageries conséquentes. A l’époque de sa création, le Zoo Circus fut un établissement qui se démarqua de la concurrence avec sa ménagerie mais également par son mode de fonctionnement. En effet, à l’époque le Zoo Circus fut le seul cirque itinérant à se déplacer via des camions alors que ses concurrents utilisaient encore le train ou les chevaux. 

La particularité de la ménagerie du Zoo Circus réside au départ plus dans l’expérience qu’on souhaite offrir au public. A ses débuts la ménagerie présentée sous une tente est modeste mais présente également des attractions et des expositions ethniques. A cette époque, le Zoo Circus ne présente pas de numéros ne présente pas encore de numéro en cage. Il fallut attendre décembre 1921, pour qu’Otto Sailer Jackson, un dresseur du Cirque Krone, fut engagé avec son groupe de tigre. Otto Sailer Jackson se présenta au Zoo Circus durant l’ensemble de la période hivernal et présenta ses tigres sous les yeux attentifs d’Alfred. Admiratif de son dresseur de fauves qui était un représentant de la méthode Hagenbeck, Alfred copia une grande majorité du style de Sailer Jackson à l’époque. Il copia jusqu’à son costume de cowboy.

En 1922, la ménagerie du Zoo Circus s’agrandit après qu’Alfred eut acheter un groupe d’ours à Carl Hagenbeck. L’année suivante il présenta alors son groupe d’ours blancs sous le nom de piste Egelton. A cette époque Alfred ne présente pas encore les fauves du Zoo Circus qui sont présenté par Martha la Corse et son mari connu comme le dompteur Marcel. Quelque année plus tard Alfred décida de louer ses numéros en cage. Ainsi fin 1924, il fit la promotion de plusieurs numéros en cage et proposa ainsi ses services aux autres cirques. Sur ce qu’on pourrait considérer comme son catalogue il proposait cinq numéros en cage tous plus étonnant les uns que les autres. Il proposa alors un groupe de dix tigres, son groupe de douze ours polaire, un groupe de dix panthères, un groupe de dix-huit lions et un groupe composé d’hyènes et de loups. 

Durant l’entre-deux-guerres, le Zoo Circus devint progressivement le cirque itinérant le plus important de France. Le succès du Zoo Circus est si grand qu’Alfred entend bien honorer la réputation de son cirque en accueillant de nombreux animaux. Ainsi en 1925, la ménagerie du Zoo Circus présentait vingt-cinq lions, sept loups, trois pumas, neuf hyènes, huit tigres, douze ours polaires, seize panthères et léopars, cinq jaguars, deux guépards ainsi qu’en grande caravane d’animaux exotique. L’envergure de la ménagerie du Zoo Circus permit à Alfred d’engager deux dresseurs : Vojtech Trubka qui présenta le groupe d’ours polaire et Johnny de Kok qui présenta son groupe de lion. C’est à cette période qu’il prit comme nom de piste Alfred Court et présenta les tigres du Zoo Circus. 

Hélas, après 1927, les problèmes financiers poussèrent les frères Court à se séparer afin diviser leur établissement. Alfred décida de monter un Zoo Circus en Espagne alors que son frère Jules prit la direction de l’enseigne en France. Le plan imaginé par les frères Court était qu’en divisant leur établissement en deux enseignes distinctes ils pourraient doubler le nombre de représentations et augmenter leur rentabilité. Ce fut un pari audacieux de la part des frères Court mais qui s’avéra plus qu’intelligent puisqu’il fut rapidement payant. L’année suivante Alfred présenta un grand numéro en cage composé de dix lions, un puma, sept tigres, sept ours bruns et ours polaire et deux grands danois.

A la fin de l’année 1928, les frères Court décidèrent de monter un cirque à trois pistes qu’ils nommèrent l’Arène Olympique à Marseille. Cependant, ce n’est pas la première fois qu’Alfred se lança dans ce type d’entreprise. L’année précédente Alfred s’associa avec Pierre Périé pour monter un cirque à trois pistes mais le programme présenté ne fut pas à la hauteur des exigences du public de l’époque et l’entreprise ferma quelque temps plus tard. Cette fois-ci, le programme du spectacle fut longuement réfléchi et l’Arène Olympique des frères Court fut un succès. En guise de numéro final, ils présentèrent au public trois numéro de cage sur chacune des pistes en même temps. Ainsi les spectacles de l’Arène Olympique se concluait sur la présentation de douze lions présentés par Vargas, de huit tigres présentés par Alfred et d’un numéro mixte de 23 bêtes présentées par Max Stolle. L’Arène Olympique recontra un grand succès dans la cité phocéenne mais dut faire face à des aléas climatiques qui provoqua sa chute. A l’approche de l’hiver, l’Arène Olympique connut une chute de fréquentation dut au froid de l’hiver. En effet à l’époque le chapiteau des frères Court ne disposait pas de système de chauffage et dû se résoudre à fermer ses portes.

Malheureusement, les péripéties d’Alfred le directeur cirque n’était pas fini. Lorsqu’il décida de remonte son Zoo Circus avec son frère Jules en 1929, une tempête de neige fit effondrer la toile et le Zoo Circus se retrouva sans chapiteau. Alfred décida donc de remonter son cirque à trois pistes par la suite afin de se refaire une trésorerie mais son spectacle n’arriva pas à séduire le public malgré des effort fait sur la publicité.
Après ces trois échecs en tant que directeur de cirque, Alfred décida louer la plupart de ses numéros d’animaux au d’autres établissement. A l’époque un nouveau Zoo Circus est monté sous la direction de son neveu, Charles Court. En 1930, le Zoo Circus repart en tournée et les frères Court décide louer le nom Wilhelm Hangenbeck et créer un nouvel établissement afin de combattre la concurrence grandissante à cette époque entre les cirques. Le Cirque Wilhelm Hangenbeck des frères Court présenta un imposant spectacle avec des animaux exotiques et plusieurs numéros en cage. La première saison du Cirque Wilhelm Hagenbeck fut un succès mais malheureusement la saison suivante ne fut pas pas à la hauteur de leur attente. En 1932, ils décidèrent donc de produire le spectacle Robison et ses tribus sauvages, un mélange d’exposition ethnique, de numéro de far west et de cirque traditionnel. Hélas ce fut encore un échec pour les frères Court, et après avoir tenté de revenir avec leur enseigne Zoo Circus ils finirent par déposer bilan en 1932.

Après cette aventure, les frères Court décidèrent se séparer et Alfred monta des numéros en cage afin de louer à d’autre établissement. Pour l’aider dans son entreprise de location de numéro il engagea Violette d’Argens qui s’occupa de son groupe de lion et de Vojtech Trubka qui se chargea de se groupe de tigre. Tandis qu’Alfred loue ses numéros en cage, son rêve de cirque ne l’a pas quitté malgré les échecs qu’il a connu. C’est ainsi qu’il décida monter le Cirque Olympia en association avec Jean Roche et Pierre Périé. Ce petit établissement tourna dans le sud de la France et en Espagne et rencontra un court succès a cause de problème de communication. En ce temps, les publicités du Cirque Olympia était trompeuse et promettait un spectacle exceptionnel alors que le spectacle présenté était modeste. La déception du public poussa alors Alfred à transformer son cirque en ménagerie forain pour la saison 1935.

Mais avant d’être un directeur de cirque, Alfred fut d’abords dresseur de fauves et après ces multiples échecs il décida de remonter un nouveau numéro en cage. Ainsi naquit La Paix dans la Jungle, un numéro mixte composé à l’époque de trois lions, trois ours polaire, trois ours brun, deux tigres et deux léopards. Dès février 1936, il présenta son numéro au Cirque Medrano, un grand cirque parisien, et se fit connaitre pour ses talents de belluaire. Alors que sa carrière de dresseur commence à fonctionner, en parallèle il continue de louer ses numéros en cage. En 1937, il monta un nouveau numéro mixte composé de de neuf léopards dont trois noirs, un léopard des neiges, un jaguar noir et 4 pumas. Avec ce groupe constitué d’espèce rarement vu au cirque mais aussi rarement présenté simultanément, Alfred rencontra un grand succès dans toute l’Europe. Il se produisit dans les plus établissement de cirque mais aussi dans les théâtres et les music-halls, notamment au Théâtre de l’Empire à Paris. Mais si Alfred profite du succès de sa jeune carrière de dresseur, l’histoire bouleversera bientôt sa carrière.
Alors qu’Alfred se produisait au Tower Circus en Angleterre, la Seconde Guerre mondiale éclata. Alfred décida d’accepter l’offre de John Ringling North et partie pour se produire pour au Ringling Bros. and Barnum \& Bailey Circus. C’est ainsi qu’il embarqua au Etats-Unis avec sa femme, ses vingt employés, ses quatre-vingts animaux et son équipement. En 1940, il signe un contrat de 2 ans, pour le grand établissement, figure sur les affiches, et présent un numéro en cage sur la piste centrale. En 1945, afin de suivre la tendance des Etats-Unis qui incluait des danseuses dans beaucoup de numéros, Alfred monta un nouveau numéro nommé La Belle et la Bête, un numéro composé de danseuses de léopards et présenté à l’époque par Willy Storey et Damoo Dhotre. 

En 1946, Alfred se faisant vieux il décida de prendre sa retraite des pistes et de rentrée avec son épouse en France. Il vendit une partie des animaux au Etats-Unis et vendit le reste de sa ménagerie aux frères Amar. Par la suite, il entreprit d’écrire ses mémoires dans sa villa à Nice. Le 30 décembre 1974, ce fut le travail de toute une carrière fut récompensé lorsqu’il reçut le premier clown d’or de l’histoire du festival du cirque Monte-Carlo. 

Alfred Court décéda le 1er juillet 1977 et repose au cimetière Caucade de Nice. 





\thispagestyle{empty} %Dernière page vide

