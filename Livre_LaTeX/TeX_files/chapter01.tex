\chapter*{Introduction}\addcontentsline{toc}{chapter}{Introduction}

Il est devenu urgent de réagir aujourd’hui plus qu’hier à ce qui pourrait s’apparenter à la mort lente, douloureuse et programmée du cirque traditionnel. En quelques années tout un héritage de culture circassienne séculaire fut remis en cause. Le cirque est ringardisé et catégorisé par une pseudo élite bobo parisienne qui, comme à l’époque du roi soleil, donne son avis sur ce qui beau ou pas, bien ou mal, désuet ou tendance… Pour ces personnes les clowns ne font plus rire mais peur, les jongleurs, quand ils sont applaudis, ne reçoivent plus les mêmes ovations populaires pour leurs prouesses, les dresseurs de fauves ne sont plus admirés pour le lien incroyable qu’ils ont réussi à tisser avec leurs animaux et l’exploit qui en découle, ils sont catalogués de bourreaux, de tortionnaires, méprisés, insultés, menacés de mort ! A les entendre les minorités aurait gagné la bataille des idées et de manière habille. Tout se fait progressivement, ce qu’ils nomment le "progrès" est fait en réalité de petites révolutions. Nous assistons tous à une période démente. Des simples bénévoles d’associations animalistes se transforment en un coup d’éclat en grands vétérinaires, comportementalistes, et pourquoi pas quand l’envie leur chante, zoologues. L’absurdité de la situation est poussée à son paroxysme quand des jeunes militants à peine âgés d’une vingtaine d’années qui, refusant de vivre avec de animaux de compagnie par éthique (antispécisme), se permettent a de donner des leçons de morale aux circassiens qui vivent avec leurs animaux depuis des générations. S’il y a bien une chose que cette époque ne craint pas c’est bien le ridicule. On a vu sur des plateaux de télévision des présidentes d’associations animalistes - qui ont d’ailleurs parfois d’animalistes que le nom, ils sont plus anti-cirque qu’autres choses - affirmant haut et fort qu’en 2021 en France on faisait sauter des fauves dans des cerceaux de feu. Depuis quand ne sont-elles pas aller voir un numéro de fauves ? Elles constateraient la grandeur et la prestance de ces animaux dans des exercices qui imitent les comportements naturels de ces grands félins dans la nature. Ce jour-là cette affirmation fallacieuse au sujet des numéros de fauves est passée inaperçue. Mais ce qui est important de noter dans cette histoire c’est en réalité la déconnexion de ses personnes avec ce qu’est devenu aujourd’hui le cirque traditionnel.
Le cirque n’est plus romain - d’ailleurs il ne l’a jamais vraiment été - il est international, pluridisciplinaire et chargé d’histoire. Il est fini le temps du dresseur qui affrontait ses fauves sur la piste tel un gladiateur, aussi appelé le dressage en "férocité". Le deuxième élément remarquable ce sont les méthodes utilisées par ces associations : mensonges, approximations manipulation des faits afin de choquer le plus grand nombre, utilisation de vidéo de cirque étranger pour stigmatiser l’ensemble des cirques français, et la liste est, hélas, encore bien longue… Qui sont les tenants et aboutissants de cette machine de propagande meurtrière qui veut la peau de notre cirque traditionnelle et de nos traditions ?  Les antispécistes, des écolos sectaires qui se servent de l’écologie pour alimenter leur idéologie et leur manière de penser. Après de longues nuits de réflexion je reconnais ne toujours pas comprendre leur but ultime mais passons… Le décor est planté. Aujourd’hui, il est plus à la mode, du moins dans le monde médiatique, de condamner le cirque que le chérir et d’en parler avec amour. Pour que ceci soit possible il faudrait que des hommes courageux, ce dont il manque cruellement à notre époque. Je n’ai pas cette prétention, et ne l’ai jamais eu d’ailleurs. 
Lorsque l’idée de créer un TikTok qui traiterait de l’histoire du cirque et de sa culture m’est venu en tête, je ne prétendais pas changer le monde, je partais simplement du constat que je n’avais jamais vu quelqu’un sur les réseaux sociaux parler du cirque avec passion et amour. Un passionné qui avait du cirque traditionnel qui coulait dans ses veines depuis 21 ans. Je voulais juste proposer ce que je ne voyais nulle part ailleurs. Avant de me lancer dans cette grande aventure il y a eu un moment d’hésitation, et sans doute la peur de ne pas trouver un public, mais j’ai senti que je devais le faire. Qui aurait eu l’audace de le faire sinon ? Voilà comment je me suis lancé sur les réseaux sociaux et c’est grâce à mon public que l’idée d’écrire un livre m’est venu. Un livre écrit par un passionné destiné aux curieux (ou curieuses) d’en apprendre plus sur ce monde merveilleux qu’est le cirque traditionnel ! Avant de démarrer ce beau voyage j’aimerais attirer votre attention sur un détail important. Le présent ouvrage n’est pas l’œuvre d’un historien ni même d’un universitaire. Mon modeste objectif est de vous donner envie de vous intéresser à cette histoire si riche et si haletante. C’est en initiant toutes mes recherches que je me suis rendu compte qu’apprendre l'histoire du cirque est en réalité est très réconfortant. J'ai donc également voulu écrire ce livre pour la jeunesse qui s'inquiète parfois de l'avenir du cirque dans la mesure où pour avoir eu la chance d'en avoir parler avec certain, leur inquiétude est grandissante, et souvent pour les rassurer je leur parle de l'histoire du cirque Pinder. Considéré pour la plupart comme une institution du cirque français, peu de personne savent que sa longue histoire est clairsemée de faillite. Pourtant il s'est toujours sorti des pires situations… Mais nous en reparlerons en plus ample détail au sein de cet ouvrage. Je vous invite donc à découvrir ma culture, séculaire par son histoire et immortel par ses traditions, de la plus belle des manières possibles, en saltimbanque que je suis.\\

\begin{center}
	Place au cirque ! 
\end{center}
\thispagestyle{empty} %Dernière page vide
